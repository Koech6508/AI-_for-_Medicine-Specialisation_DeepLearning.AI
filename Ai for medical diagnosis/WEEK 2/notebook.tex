
% Default to the notebook output style

    


% Inherit from the specified cell style.




    
\documentclass[11pt]{article}

    
    
    \usepackage[T1]{fontenc}
    % Nicer default font (+ math font) than Computer Modern for most use cases
    \usepackage{mathpazo}

    % Basic figure setup, for now with no caption control since it's done
    % automatically by Pandoc (which extracts ![](path) syntax from Markdown).
    \usepackage{graphicx}
    % We will generate all images so they have a width \maxwidth. This means
    % that they will get their normal width if they fit onto the page, but
    % are scaled down if they would overflow the margins.
    \makeatletter
    \def\maxwidth{\ifdim\Gin@nat@width>\linewidth\linewidth
    \else\Gin@nat@width\fi}
    \makeatother
    \let\Oldincludegraphics\includegraphics
    % Set max figure width to be 80% of text width, for now hardcoded.
    \renewcommand{\includegraphics}[1]{\Oldincludegraphics[width=.8\maxwidth]{#1}}
    % Ensure that by default, figures have no caption (until we provide a
    % proper Figure object with a Caption API and a way to capture that
    % in the conversion process - todo).
    \usepackage{caption}
    \DeclareCaptionLabelFormat{nolabel}{}
    \captionsetup{labelformat=nolabel}

    \usepackage{adjustbox} % Used to constrain images to a maximum size 
    \usepackage{xcolor} % Allow colors to be defined
    \usepackage{enumerate} % Needed for markdown enumerations to work
    \usepackage{geometry} % Used to adjust the document margins
    \usepackage{amsmath} % Equations
    \usepackage{amssymb} % Equations
    \usepackage{textcomp} % defines textquotesingle
    % Hack from http://tex.stackexchange.com/a/47451/13684:
    \AtBeginDocument{%
        \def\PYZsq{\textquotesingle}% Upright quotes in Pygmentized code
    }
    \usepackage{upquote} % Upright quotes for verbatim code
    \usepackage{eurosym} % defines \euro
    \usepackage[mathletters]{ucs} % Extended unicode (utf-8) support
    \usepackage[utf8x]{inputenc} % Allow utf-8 characters in the tex document
    \usepackage{fancyvrb} % verbatim replacement that allows latex
    \usepackage{grffile} % extends the file name processing of package graphics 
                         % to support a larger range 
    % The hyperref package gives us a pdf with properly built
    % internal navigation ('pdf bookmarks' for the table of contents,
    % internal cross-reference links, web links for URLs, etc.)
    \usepackage{hyperref}
    \usepackage{longtable} % longtable support required by pandoc >1.10
    \usepackage{booktabs}  % table support for pandoc > 1.12.2
    \usepackage[inline]{enumitem} % IRkernel/repr support (it uses the enumerate* environment)
    \usepackage[normalem]{ulem} % ulem is needed to support strikethroughs (\sout)
                                % normalem makes italics be italics, not underlines
    

    
    
    % Colors for the hyperref package
    \definecolor{urlcolor}{rgb}{0,.145,.698}
    \definecolor{linkcolor}{rgb}{.71,0.21,0.01}
    \definecolor{citecolor}{rgb}{.12,.54,.11}

    % ANSI colors
    \definecolor{ansi-black}{HTML}{3E424D}
    \definecolor{ansi-black-intense}{HTML}{282C36}
    \definecolor{ansi-red}{HTML}{E75C58}
    \definecolor{ansi-red-intense}{HTML}{B22B31}
    \definecolor{ansi-green}{HTML}{00A250}
    \definecolor{ansi-green-intense}{HTML}{007427}
    \definecolor{ansi-yellow}{HTML}{DDB62B}
    \definecolor{ansi-yellow-intense}{HTML}{B27D12}
    \definecolor{ansi-blue}{HTML}{208FFB}
    \definecolor{ansi-blue-intense}{HTML}{0065CA}
    \definecolor{ansi-magenta}{HTML}{D160C4}
    \definecolor{ansi-magenta-intense}{HTML}{A03196}
    \definecolor{ansi-cyan}{HTML}{60C6C8}
    \definecolor{ansi-cyan-intense}{HTML}{258F8F}
    \definecolor{ansi-white}{HTML}{C5C1B4}
    \definecolor{ansi-white-intense}{HTML}{A1A6B2}

    % commands and environments needed by pandoc snippets
    % extracted from the output of `pandoc -s`
    \providecommand{\tightlist}{%
      \setlength{\itemsep}{0pt}\setlength{\parskip}{0pt}}
    \DefineVerbatimEnvironment{Highlighting}{Verbatim}{commandchars=\\\{\}}
    % Add ',fontsize=\small' for more characters per line
    \newenvironment{Shaded}{}{}
    \newcommand{\KeywordTok}[1]{\textcolor[rgb]{0.00,0.44,0.13}{\textbf{{#1}}}}
    \newcommand{\DataTypeTok}[1]{\textcolor[rgb]{0.56,0.13,0.00}{{#1}}}
    \newcommand{\DecValTok}[1]{\textcolor[rgb]{0.25,0.63,0.44}{{#1}}}
    \newcommand{\BaseNTok}[1]{\textcolor[rgb]{0.25,0.63,0.44}{{#1}}}
    \newcommand{\FloatTok}[1]{\textcolor[rgb]{0.25,0.63,0.44}{{#1}}}
    \newcommand{\CharTok}[1]{\textcolor[rgb]{0.25,0.44,0.63}{{#1}}}
    \newcommand{\StringTok}[1]{\textcolor[rgb]{0.25,0.44,0.63}{{#1}}}
    \newcommand{\CommentTok}[1]{\textcolor[rgb]{0.38,0.63,0.69}{\textit{{#1}}}}
    \newcommand{\OtherTok}[1]{\textcolor[rgb]{0.00,0.44,0.13}{{#1}}}
    \newcommand{\AlertTok}[1]{\textcolor[rgb]{1.00,0.00,0.00}{\textbf{{#1}}}}
    \newcommand{\FunctionTok}[1]{\textcolor[rgb]{0.02,0.16,0.49}{{#1}}}
    \newcommand{\RegionMarkerTok}[1]{{#1}}
    \newcommand{\ErrorTok}[1]{\textcolor[rgb]{1.00,0.00,0.00}{\textbf{{#1}}}}
    \newcommand{\NormalTok}[1]{{#1}}
    
    % Additional commands for more recent versions of Pandoc
    \newcommand{\ConstantTok}[1]{\textcolor[rgb]{0.53,0.00,0.00}{{#1}}}
    \newcommand{\SpecialCharTok}[1]{\textcolor[rgb]{0.25,0.44,0.63}{{#1}}}
    \newcommand{\VerbatimStringTok}[1]{\textcolor[rgb]{0.25,0.44,0.63}{{#1}}}
    \newcommand{\SpecialStringTok}[1]{\textcolor[rgb]{0.73,0.40,0.53}{{#1}}}
    \newcommand{\ImportTok}[1]{{#1}}
    \newcommand{\DocumentationTok}[1]{\textcolor[rgb]{0.73,0.13,0.13}{\textit{{#1}}}}
    \newcommand{\AnnotationTok}[1]{\textcolor[rgb]{0.38,0.63,0.69}{\textbf{\textit{{#1}}}}}
    \newcommand{\CommentVarTok}[1]{\textcolor[rgb]{0.38,0.63,0.69}{\textbf{\textit{{#1}}}}}
    \newcommand{\VariableTok}[1]{\textcolor[rgb]{0.10,0.09,0.49}{{#1}}}
    \newcommand{\ControlFlowTok}[1]{\textcolor[rgb]{0.00,0.44,0.13}{\textbf{{#1}}}}
    \newcommand{\OperatorTok}[1]{\textcolor[rgb]{0.40,0.40,0.40}{{#1}}}
    \newcommand{\BuiltInTok}[1]{{#1}}
    \newcommand{\ExtensionTok}[1]{{#1}}
    \newcommand{\PreprocessorTok}[1]{\textcolor[rgb]{0.74,0.48,0.00}{{#1}}}
    \newcommand{\AttributeTok}[1]{\textcolor[rgb]{0.49,0.56,0.16}{{#1}}}
    \newcommand{\InformationTok}[1]{\textcolor[rgb]{0.38,0.63,0.69}{\textbf{\textit{{#1}}}}}
    \newcommand{\WarningTok}[1]{\textcolor[rgb]{0.38,0.63,0.69}{\textbf{\textit{{#1}}}}}
    
    
    % Define a nice break command that doesn't care if a line doesn't already
    % exist.
    \def\br{\hspace*{\fill} \\* }
    % Math Jax compatability definitions
    \def\gt{>}
    \def\lt{<}
    % Document parameters
    \title{C1\_W2\_Assignment}
    
    
    

    % Pygments definitions
    
\makeatletter
\def\PY@reset{\let\PY@it=\relax \let\PY@bf=\relax%
    \let\PY@ul=\relax \let\PY@tc=\relax%
    \let\PY@bc=\relax \let\PY@ff=\relax}
\def\PY@tok#1{\csname PY@tok@#1\endcsname}
\def\PY@toks#1+{\ifx\relax#1\empty\else%
    \PY@tok{#1}\expandafter\PY@toks\fi}
\def\PY@do#1{\PY@bc{\PY@tc{\PY@ul{%
    \PY@it{\PY@bf{\PY@ff{#1}}}}}}}
\def\PY#1#2{\PY@reset\PY@toks#1+\relax+\PY@do{#2}}

\expandafter\def\csname PY@tok@w\endcsname{\def\PY@tc##1{\textcolor[rgb]{0.73,0.73,0.73}{##1}}}
\expandafter\def\csname PY@tok@c\endcsname{\let\PY@it=\textit\def\PY@tc##1{\textcolor[rgb]{0.25,0.50,0.50}{##1}}}
\expandafter\def\csname PY@tok@cp\endcsname{\def\PY@tc##1{\textcolor[rgb]{0.74,0.48,0.00}{##1}}}
\expandafter\def\csname PY@tok@k\endcsname{\let\PY@bf=\textbf\def\PY@tc##1{\textcolor[rgb]{0.00,0.50,0.00}{##1}}}
\expandafter\def\csname PY@tok@kp\endcsname{\def\PY@tc##1{\textcolor[rgb]{0.00,0.50,0.00}{##1}}}
\expandafter\def\csname PY@tok@kt\endcsname{\def\PY@tc##1{\textcolor[rgb]{0.69,0.00,0.25}{##1}}}
\expandafter\def\csname PY@tok@o\endcsname{\def\PY@tc##1{\textcolor[rgb]{0.40,0.40,0.40}{##1}}}
\expandafter\def\csname PY@tok@ow\endcsname{\let\PY@bf=\textbf\def\PY@tc##1{\textcolor[rgb]{0.67,0.13,1.00}{##1}}}
\expandafter\def\csname PY@tok@nb\endcsname{\def\PY@tc##1{\textcolor[rgb]{0.00,0.50,0.00}{##1}}}
\expandafter\def\csname PY@tok@nf\endcsname{\def\PY@tc##1{\textcolor[rgb]{0.00,0.00,1.00}{##1}}}
\expandafter\def\csname PY@tok@nc\endcsname{\let\PY@bf=\textbf\def\PY@tc##1{\textcolor[rgb]{0.00,0.00,1.00}{##1}}}
\expandafter\def\csname PY@tok@nn\endcsname{\let\PY@bf=\textbf\def\PY@tc##1{\textcolor[rgb]{0.00,0.00,1.00}{##1}}}
\expandafter\def\csname PY@tok@ne\endcsname{\let\PY@bf=\textbf\def\PY@tc##1{\textcolor[rgb]{0.82,0.25,0.23}{##1}}}
\expandafter\def\csname PY@tok@nv\endcsname{\def\PY@tc##1{\textcolor[rgb]{0.10,0.09,0.49}{##1}}}
\expandafter\def\csname PY@tok@no\endcsname{\def\PY@tc##1{\textcolor[rgb]{0.53,0.00,0.00}{##1}}}
\expandafter\def\csname PY@tok@nl\endcsname{\def\PY@tc##1{\textcolor[rgb]{0.63,0.63,0.00}{##1}}}
\expandafter\def\csname PY@tok@ni\endcsname{\let\PY@bf=\textbf\def\PY@tc##1{\textcolor[rgb]{0.60,0.60,0.60}{##1}}}
\expandafter\def\csname PY@tok@na\endcsname{\def\PY@tc##1{\textcolor[rgb]{0.49,0.56,0.16}{##1}}}
\expandafter\def\csname PY@tok@nt\endcsname{\let\PY@bf=\textbf\def\PY@tc##1{\textcolor[rgb]{0.00,0.50,0.00}{##1}}}
\expandafter\def\csname PY@tok@nd\endcsname{\def\PY@tc##1{\textcolor[rgb]{0.67,0.13,1.00}{##1}}}
\expandafter\def\csname PY@tok@s\endcsname{\def\PY@tc##1{\textcolor[rgb]{0.73,0.13,0.13}{##1}}}
\expandafter\def\csname PY@tok@sd\endcsname{\let\PY@it=\textit\def\PY@tc##1{\textcolor[rgb]{0.73,0.13,0.13}{##1}}}
\expandafter\def\csname PY@tok@si\endcsname{\let\PY@bf=\textbf\def\PY@tc##1{\textcolor[rgb]{0.73,0.40,0.53}{##1}}}
\expandafter\def\csname PY@tok@se\endcsname{\let\PY@bf=\textbf\def\PY@tc##1{\textcolor[rgb]{0.73,0.40,0.13}{##1}}}
\expandafter\def\csname PY@tok@sr\endcsname{\def\PY@tc##1{\textcolor[rgb]{0.73,0.40,0.53}{##1}}}
\expandafter\def\csname PY@tok@ss\endcsname{\def\PY@tc##1{\textcolor[rgb]{0.10,0.09,0.49}{##1}}}
\expandafter\def\csname PY@tok@sx\endcsname{\def\PY@tc##1{\textcolor[rgb]{0.00,0.50,0.00}{##1}}}
\expandafter\def\csname PY@tok@m\endcsname{\def\PY@tc##1{\textcolor[rgb]{0.40,0.40,0.40}{##1}}}
\expandafter\def\csname PY@tok@gh\endcsname{\let\PY@bf=\textbf\def\PY@tc##1{\textcolor[rgb]{0.00,0.00,0.50}{##1}}}
\expandafter\def\csname PY@tok@gu\endcsname{\let\PY@bf=\textbf\def\PY@tc##1{\textcolor[rgb]{0.50,0.00,0.50}{##1}}}
\expandafter\def\csname PY@tok@gd\endcsname{\def\PY@tc##1{\textcolor[rgb]{0.63,0.00,0.00}{##1}}}
\expandafter\def\csname PY@tok@gi\endcsname{\def\PY@tc##1{\textcolor[rgb]{0.00,0.63,0.00}{##1}}}
\expandafter\def\csname PY@tok@gr\endcsname{\def\PY@tc##1{\textcolor[rgb]{1.00,0.00,0.00}{##1}}}
\expandafter\def\csname PY@tok@ge\endcsname{\let\PY@it=\textit}
\expandafter\def\csname PY@tok@gs\endcsname{\let\PY@bf=\textbf}
\expandafter\def\csname PY@tok@gp\endcsname{\let\PY@bf=\textbf\def\PY@tc##1{\textcolor[rgb]{0.00,0.00,0.50}{##1}}}
\expandafter\def\csname PY@tok@go\endcsname{\def\PY@tc##1{\textcolor[rgb]{0.53,0.53,0.53}{##1}}}
\expandafter\def\csname PY@tok@gt\endcsname{\def\PY@tc##1{\textcolor[rgb]{0.00,0.27,0.87}{##1}}}
\expandafter\def\csname PY@tok@err\endcsname{\def\PY@bc##1{\setlength{\fboxsep}{0pt}\fcolorbox[rgb]{1.00,0.00,0.00}{1,1,1}{\strut ##1}}}
\expandafter\def\csname PY@tok@kc\endcsname{\let\PY@bf=\textbf\def\PY@tc##1{\textcolor[rgb]{0.00,0.50,0.00}{##1}}}
\expandafter\def\csname PY@tok@kd\endcsname{\let\PY@bf=\textbf\def\PY@tc##1{\textcolor[rgb]{0.00,0.50,0.00}{##1}}}
\expandafter\def\csname PY@tok@kn\endcsname{\let\PY@bf=\textbf\def\PY@tc##1{\textcolor[rgb]{0.00,0.50,0.00}{##1}}}
\expandafter\def\csname PY@tok@kr\endcsname{\let\PY@bf=\textbf\def\PY@tc##1{\textcolor[rgb]{0.00,0.50,0.00}{##1}}}
\expandafter\def\csname PY@tok@bp\endcsname{\def\PY@tc##1{\textcolor[rgb]{0.00,0.50,0.00}{##1}}}
\expandafter\def\csname PY@tok@fm\endcsname{\def\PY@tc##1{\textcolor[rgb]{0.00,0.00,1.00}{##1}}}
\expandafter\def\csname PY@tok@vc\endcsname{\def\PY@tc##1{\textcolor[rgb]{0.10,0.09,0.49}{##1}}}
\expandafter\def\csname PY@tok@vg\endcsname{\def\PY@tc##1{\textcolor[rgb]{0.10,0.09,0.49}{##1}}}
\expandafter\def\csname PY@tok@vi\endcsname{\def\PY@tc##1{\textcolor[rgb]{0.10,0.09,0.49}{##1}}}
\expandafter\def\csname PY@tok@vm\endcsname{\def\PY@tc##1{\textcolor[rgb]{0.10,0.09,0.49}{##1}}}
\expandafter\def\csname PY@tok@sa\endcsname{\def\PY@tc##1{\textcolor[rgb]{0.73,0.13,0.13}{##1}}}
\expandafter\def\csname PY@tok@sb\endcsname{\def\PY@tc##1{\textcolor[rgb]{0.73,0.13,0.13}{##1}}}
\expandafter\def\csname PY@tok@sc\endcsname{\def\PY@tc##1{\textcolor[rgb]{0.73,0.13,0.13}{##1}}}
\expandafter\def\csname PY@tok@dl\endcsname{\def\PY@tc##1{\textcolor[rgb]{0.73,0.13,0.13}{##1}}}
\expandafter\def\csname PY@tok@s2\endcsname{\def\PY@tc##1{\textcolor[rgb]{0.73,0.13,0.13}{##1}}}
\expandafter\def\csname PY@tok@sh\endcsname{\def\PY@tc##1{\textcolor[rgb]{0.73,0.13,0.13}{##1}}}
\expandafter\def\csname PY@tok@s1\endcsname{\def\PY@tc##1{\textcolor[rgb]{0.73,0.13,0.13}{##1}}}
\expandafter\def\csname PY@tok@mb\endcsname{\def\PY@tc##1{\textcolor[rgb]{0.40,0.40,0.40}{##1}}}
\expandafter\def\csname PY@tok@mf\endcsname{\def\PY@tc##1{\textcolor[rgb]{0.40,0.40,0.40}{##1}}}
\expandafter\def\csname PY@tok@mh\endcsname{\def\PY@tc##1{\textcolor[rgb]{0.40,0.40,0.40}{##1}}}
\expandafter\def\csname PY@tok@mi\endcsname{\def\PY@tc##1{\textcolor[rgb]{0.40,0.40,0.40}{##1}}}
\expandafter\def\csname PY@tok@il\endcsname{\def\PY@tc##1{\textcolor[rgb]{0.40,0.40,0.40}{##1}}}
\expandafter\def\csname PY@tok@mo\endcsname{\def\PY@tc##1{\textcolor[rgb]{0.40,0.40,0.40}{##1}}}
\expandafter\def\csname PY@tok@ch\endcsname{\let\PY@it=\textit\def\PY@tc##1{\textcolor[rgb]{0.25,0.50,0.50}{##1}}}
\expandafter\def\csname PY@tok@cm\endcsname{\let\PY@it=\textit\def\PY@tc##1{\textcolor[rgb]{0.25,0.50,0.50}{##1}}}
\expandafter\def\csname PY@tok@cpf\endcsname{\let\PY@it=\textit\def\PY@tc##1{\textcolor[rgb]{0.25,0.50,0.50}{##1}}}
\expandafter\def\csname PY@tok@c1\endcsname{\let\PY@it=\textit\def\PY@tc##1{\textcolor[rgb]{0.25,0.50,0.50}{##1}}}
\expandafter\def\csname PY@tok@cs\endcsname{\let\PY@it=\textit\def\PY@tc##1{\textcolor[rgb]{0.25,0.50,0.50}{##1}}}

\def\PYZbs{\char`\\}
\def\PYZus{\char`\_}
\def\PYZob{\char`\{}
\def\PYZcb{\char`\}}
\def\PYZca{\char`\^}
\def\PYZam{\char`\&}
\def\PYZlt{\char`\<}
\def\PYZgt{\char`\>}
\def\PYZsh{\char`\#}
\def\PYZpc{\char`\%}
\def\PYZdl{\char`\$}
\def\PYZhy{\char`\-}
\def\PYZsq{\char`\'}
\def\PYZdq{\char`\"}
\def\PYZti{\char`\~}
% for compatibility with earlier versions
\def\PYZat{@}
\def\PYZlb{[}
\def\PYZrb{]}
\makeatother


    % Exact colors from NB
    \definecolor{incolor}{rgb}{0.0, 0.0, 0.5}
    \definecolor{outcolor}{rgb}{0.545, 0.0, 0.0}



    
    % Prevent overflowing lines due to hard-to-break entities
    \sloppy 
    % Setup hyperref package
    \hypersetup{
      breaklinks=true,  % so long urls are correctly broken across lines
      colorlinks=true,
      urlcolor=urlcolor,
      linkcolor=linkcolor,
      citecolor=citecolor,
      }
    % Slightly bigger margins than the latex defaults
    
    \geometry{verbose,tmargin=1in,bmargin=1in,lmargin=1in,rmargin=1in}
    
    

    \begin{document}
    
    
    \maketitle
    
    

    
    \hypertarget{evaluation-of-diagnostic-models}{%
\section{Evaluation of Diagnostic
Models}\label{evaluation-of-diagnostic-models}}

Welcome to the second assignment of course 1. In this assignment, we
will be working with the results of the X-ray classification model we
developed in the previous assignment. In order to make the data
processing a bit more manageable, we will be working with a subset of
our training, and validation datasets. We will also use our manually
labeled test dataset of 420 X-rays.

As a reminder, our dataset contains X-rays from 14 different conditions
diagnosable from an X-ray. We'll evaluate our performance on each of
these classes using the classification metrics we learned in lecture.

\textbf{By the end of this assignment you will learn about:}

\begin{enumerate}
\def\labelenumi{\arabic{enumi}.}
\tightlist
\item
  Accuracy
\item
  Prevalence
\item
  Specificity \& Sensitivity
\item
  PPV and NPV
\item
  ROC curve and AUCROC (c-statistic)
\item
  Confidence Intervals
\end{enumerate}

    \hypertarget{table-of-contents}{%
\subsection{Table of Contents}\label{table-of-contents}}

\begin{itemize}
\tightlist
\item
  Section \ref{1}
\item
  Section \ref{2}
\item
  Section \ref{3}

  \begin{itemize}
  \tightlist
  \item
    Section \ref{3-1}

    \begin{itemize}
    \tightlist
    \item
      Section \ref{ex-1}
    \end{itemize}
  \item
    Section \ref{3-2}

    \begin{itemize}
    \tightlist
    \item
      Section \ref{ex-2}
    \end{itemize}
  \item
    Section \ref{3-3}

    \begin{itemize}
    \tightlist
    \item
      Section \ref{ex-3}
    \end{itemize}
  \item
    Section \ref{3-4}

    \begin{itemize}
    \tightlist
    \item
      Section \ref{ex-4}
    \end{itemize}
  \item
    Section \ref{3-5}

    \begin{itemize}
    \tightlist
    \item
      Section \ref{ex-5}
    \end{itemize}
  \item
    Section \ref{3-6}
  \end{itemize}
\item
  Section \ref{4}
\item
  Section \ref{5}
\item
  Section \ref{6}
\item
  Section \ref{7}
\end{itemize}

     \#\# 1. Packages

In this assignment, we'll make use of the following packages: -
\href{https://docs.scipy.org/doc/numpy/}{numpy} is a popular library for
scientific computing -
\href{https://matplotlib.org/3.1.1/contents.html}{matplotlib} is a
plotting library compatible with numpy -
\href{https://pandas.pydata.org/docs/}{pandas} is what we'll use to
manipulate our data -
\href{https://scikit-learn.org/stable/index.html}{sklearn} will be used
to measure the performance of our model

Run the next cell to import all the necessary packages as well as custom
util functions.

    \begin{Verbatim}[commandchars=\\\{\}]
{\color{incolor}In [{\color{incolor}1}]:} \PY{k+kn}{import} \PY{n+nn}{numpy} \PY{k}{as} \PY{n+nn}{np}
        \PY{k+kn}{import} \PY{n+nn}{matplotlib}\PY{n+nn}{.}\PY{n+nn}{pyplot} \PY{k}{as} \PY{n+nn}{plt} 
        \PY{k+kn}{import} \PY{n+nn}{pandas} \PY{k}{as} \PY{n+nn}{pd}  
        
        \PY{k+kn}{import} \PY{n+nn}{util}
        \PY{k+kn}{from} \PY{n+nn}{public\PYZus{}tests} \PY{k}{import} \PY{o}{*}
        \PY{k+kn}{from} \PY{n+nn}{test\PYZus{}utils} \PY{k}{import} \PY{o}{*}
\end{Verbatim}


     \#\# 2. Overview

We'll go through our evaluation metrics in the following order.

\begin{itemize}
\tightlist
\item
  Metrics

  \begin{itemize}
  \tightlist
  \item
    TP, TN, FP, FN
  \item
    Accuracy
  \item
    Prevalence
  \item
    Sensitivity and Specificity
  \item
    PPV and NPV
  \item
    AUC
  \end{itemize}
\item
  Confidence Intervals
\end{itemize}

    Let's take a quick peek at our dataset. The data is stored in two CSV
files called \texttt{train\_preds.csv} and \texttt{valid\_preds.csv}. We
have precomputed the model outputs for our test cases. We'll work with
these predictions and the true class labels throughout the assignment.

    \begin{Verbatim}[commandchars=\\\{\}]
{\color{incolor}In [{\color{incolor}2}]:} \PY{n}{train\PYZus{}results} \PY{o}{=} \PY{n}{pd}\PY{o}{.}\PY{n}{read\PYZus{}csv}\PY{p}{(}\PY{l+s+s2}{\PYZdq{}}\PY{l+s+s2}{data/train\PYZus{}preds.csv}\PY{l+s+s2}{\PYZdq{}}\PY{p}{)}
        \PY{n}{valid\PYZus{}results} \PY{o}{=} \PY{n}{pd}\PY{o}{.}\PY{n}{read\PYZus{}csv}\PY{p}{(}\PY{l+s+s2}{\PYZdq{}}\PY{l+s+s2}{data/valid\PYZus{}preds.csv}\PY{l+s+s2}{\PYZdq{}}\PY{p}{)}
        
        \PY{c+c1}{\PYZsh{} the labels in our dataset}
        \PY{n}{class\PYZus{}labels} \PY{o}{=} \PY{p}{[}\PY{l+s+s1}{\PYZsq{}}\PY{l+s+s1}{Cardiomegaly}\PY{l+s+s1}{\PYZsq{}}\PY{p}{,}
         \PY{l+s+s1}{\PYZsq{}}\PY{l+s+s1}{Emphysema}\PY{l+s+s1}{\PYZsq{}}\PY{p}{,}
         \PY{l+s+s1}{\PYZsq{}}\PY{l+s+s1}{Effusion}\PY{l+s+s1}{\PYZsq{}}\PY{p}{,}
         \PY{l+s+s1}{\PYZsq{}}\PY{l+s+s1}{Hernia}\PY{l+s+s1}{\PYZsq{}}\PY{p}{,}
         \PY{l+s+s1}{\PYZsq{}}\PY{l+s+s1}{Infiltration}\PY{l+s+s1}{\PYZsq{}}\PY{p}{,}
         \PY{l+s+s1}{\PYZsq{}}\PY{l+s+s1}{Mass}\PY{l+s+s1}{\PYZsq{}}\PY{p}{,}
         \PY{l+s+s1}{\PYZsq{}}\PY{l+s+s1}{Nodule}\PY{l+s+s1}{\PYZsq{}}\PY{p}{,}
         \PY{l+s+s1}{\PYZsq{}}\PY{l+s+s1}{Atelectasis}\PY{l+s+s1}{\PYZsq{}}\PY{p}{,}
         \PY{l+s+s1}{\PYZsq{}}\PY{l+s+s1}{Pneumothorax}\PY{l+s+s1}{\PYZsq{}}\PY{p}{,}
         \PY{l+s+s1}{\PYZsq{}}\PY{l+s+s1}{Pleural\PYZus{}Thickening}\PY{l+s+s1}{\PYZsq{}}\PY{p}{,}
         \PY{l+s+s1}{\PYZsq{}}\PY{l+s+s1}{Pneumonia}\PY{l+s+s1}{\PYZsq{}}\PY{p}{,}
         \PY{l+s+s1}{\PYZsq{}}\PY{l+s+s1}{Fibrosis}\PY{l+s+s1}{\PYZsq{}}\PY{p}{,}
         \PY{l+s+s1}{\PYZsq{}}\PY{l+s+s1}{Edema}\PY{l+s+s1}{\PYZsq{}}\PY{p}{,}
         \PY{l+s+s1}{\PYZsq{}}\PY{l+s+s1}{Consolidation}\PY{l+s+s1}{\PYZsq{}}\PY{p}{]}
        
        \PY{c+c1}{\PYZsh{} the labels for prediction values in our dataset}
        \PY{n}{pred\PYZus{}labels} \PY{o}{=} \PY{p}{[}\PY{n}{l} \PY{o}{+} \PY{l+s+s2}{\PYZdq{}}\PY{l+s+s2}{\PYZus{}pred}\PY{l+s+s2}{\PYZdq{}} \PY{k}{for} \PY{n}{l} \PY{o+ow}{in} \PY{n}{class\PYZus{}labels}\PY{p}{]}
\end{Verbatim}


    Extract the labels (y) and the predictions (pred).

    \begin{Verbatim}[commandchars=\\\{\}]
{\color{incolor}In [{\color{incolor}3}]:} \PY{n}{y} \PY{o}{=} \PY{n}{valid\PYZus{}results}\PY{p}{[}\PY{n}{class\PYZus{}labels}\PY{p}{]}\PY{o}{.}\PY{n}{values}
        \PY{n}{pred} \PY{o}{=} \PY{n}{valid\PYZus{}results}\PY{p}{[}\PY{n}{pred\PYZus{}labels}\PY{p}{]}\PY{o}{.}\PY{n}{values}
\end{Verbatim}


    Run the next cell to view them side by side.

    \begin{Verbatim}[commandchars=\\\{\}]
{\color{incolor}In [{\color{incolor}4}]:} \PY{c+c1}{\PYZsh{} let\PYZsq{}s take a peek at our dataset}
        \PY{n}{valid\PYZus{}results}\PY{p}{[}\PY{n}{np}\PY{o}{.}\PY{n}{concatenate}\PY{p}{(}\PY{p}{[}\PY{n}{class\PYZus{}labels}\PY{p}{,} \PY{n}{pred\PYZus{}labels}\PY{p}{]}\PY{p}{)}\PY{p}{]}\PY{o}{.}\PY{n}{head}\PY{p}{(}\PY{p}{)}
\end{Verbatim}


\begin{Verbatim}[commandchars=\\\{\}]
{\color{outcolor}Out[{\color{outcolor}4}]:}    Cardiomegaly  Emphysema  Effusion  Hernia  Infiltration  Mass  Nodule  \textbackslash{}
        0             0          0         0       0             0     0       0   
        1             0          0         0       0             1     0       1   
        2             0          0         0       0             0     0       0   
        3             0          0         0       0             0     0       0   
        4             0          0         0       0             0     0       0   
        
           Atelectasis  Pneumothorax  Pleural\_Thickening  {\ldots}  Infiltration\_pred  \textbackslash{}
        0            0             0                   0  {\ldots}           0.256020   
        1            0             0                   0  {\ldots}           0.382199   
        2            0             0                   0  {\ldots}           0.427727   
        3            0             0                   0  {\ldots}           0.158596   
        4            0             0                   0  {\ldots}           0.536762   
        
           Mass\_pred  Nodule\_pred  Atelectasis\_pred  Pneumothorax\_pred  \textbackslash{}
        0   0.266928     0.312440          0.460342           0.079453   
        1   0.176825     0.465807          0.489424           0.084595   
        2   0.115513     0.249030          0.035105           0.238761   
        3   0.259460     0.334870          0.266489           0.073371   
        4   0.198797     0.273110          0.186771           0.242122   
        
           Pleural\_Thickening\_pred  Pneumonia\_pred  Fibrosis\_pred  Edema\_pred  \textbackslash{}
        0                 0.271495        0.276861       0.398799    0.015867   
        1                 0.377318        0.363582       0.638024    0.025948   
        2                 0.167095        0.166389       0.262463    0.007758   
        3                 0.229834        0.191281       0.344348    0.008559   
        4                 0.309786        0.411771       0.244666    0.126930   
        
           Consolidation\_pred  
        0            0.156320  
        1            0.144419  
        2            0.125790  
        3            0.119153  
        4            0.342409  
        
        [5 rows x 28 columns]
\end{Verbatim}
            
    To further understand our dataset details, here's a histogram of the
number of samples for each label in the validation dataset:

    \begin{Verbatim}[commandchars=\\\{\}]
{\color{incolor}In [{\color{incolor}5}]:} \PY{n}{plt}\PY{o}{.}\PY{n}{xticks}\PY{p}{(}\PY{n}{rotation}\PY{o}{=}\PY{l+m+mi}{90}\PY{p}{)}
        \PY{n}{plt}\PY{o}{.}\PY{n}{bar}\PY{p}{(}\PY{n}{x} \PY{o}{=} \PY{n}{class\PYZus{}labels}\PY{p}{,} \PY{n}{height}\PY{o}{=} \PY{n}{y}\PY{o}{.}\PY{n}{sum}\PY{p}{(}\PY{n}{axis}\PY{o}{=}\PY{l+m+mi}{0}\PY{p}{)}\PY{p}{)}\PY{p}{;}
\end{Verbatim}


    \begin{center}
    \adjustimage{max size={0.9\linewidth}{0.9\paperheight}}{output_12_0.png}
    \end{center}
    { \hspace*{\fill} \\}
    
    It seem like our dataset has an imbalanced population of samples.
Specifically, our dataset has a small number of patients diagnosed with
a \texttt{Hernia}.

     \#\# 3. Metrics

 \#\#\# 3.1 True Positives, False Positives, True Negatives and False
Negatives

The most basic statistics to compute from the model predictions are the
true positives, true negatives, false positives, and false negatives.

As the name suggests - True Positive (TP): The model classifies the
example as positive, and the actual label also positive. - False
Positive (FP): The model classifies the example as positive,
\textbf{but} the actual label is negative. - True Negative (TN): The
model classifies the example as negative, and the actual label is also
negative. - False Negative (FN): The model classifies the example as
negative, \textbf{but} the label is actually positive.

We will count the number of TP, FP, TN and FN in the given data. All of
our metrics can be built off of these four statistics.

Recall that the model outputs real numbers between 0 and 1. * To compute
binary class predictions, we need to convert these to either 0 or 1. *
We'll do this using a threshold value \(th\). * Any model outputs above
\(th\) are set to 1, and below \(th\) are set to 0.

All of our metrics (except for AUC at the end) will depend on the choice
of this threshold.

 \#\#\# Exercise 1 - true positives, false positives, true negatives and
false negatives

Fill in the functions to compute the TP, FP, TN, and FN for a given
threshold below.

The first one has been done for you.

    \begin{Verbatim}[commandchars=\\\{\}]
{\color{incolor}In [{\color{incolor}14}]:} \PY{c+c1}{\PYZsh{} UNQ\PYZus{}C1 (UNIQUE CELL IDENTIFIER, DO NOT EDIT)}
         \PY{k}{def} \PY{n+nf}{true\PYZus{}positives}\PY{p}{(}\PY{n}{y}\PY{p}{,} \PY{n}{pred}\PY{p}{,} \PY{n}{th}\PY{o}{=}\PY{l+m+mf}{0.5}\PY{p}{)}\PY{p}{:}
             \PY{l+s+sd}{\PYZdq{}\PYZdq{}\PYZdq{}}
         \PY{l+s+sd}{    Count true positives.}
         
         \PY{l+s+sd}{    Args:}
         \PY{l+s+sd}{        y (np.array): ground truth, size (n\PYZus{}examples)}
         \PY{l+s+sd}{        pred (np.array): model output, size (n\PYZus{}examples)}
         \PY{l+s+sd}{        th (float): cutoff value for positive prediction from model}
         \PY{l+s+sd}{    Returns:}
         \PY{l+s+sd}{        TP (int): true positives}
         \PY{l+s+sd}{    \PYZdq{}\PYZdq{}\PYZdq{}}
             \PY{n}{TP} \PY{o}{=} \PY{l+m+mi}{0}
             
             \PY{c+c1}{\PYZsh{} get thresholded predictions}
             \PY{n}{thresholded\PYZus{}preds} \PY{o}{=} \PY{n}{pred} \PY{o}{\PYZgt{}}\PY{o}{=} \PY{n}{th}
         
             \PY{c+c1}{\PYZsh{} compute TP}
             \PY{n}{TP} \PY{o}{=} \PY{n}{np}\PY{o}{.}\PY{n}{sum}\PY{p}{(}\PY{p}{(}\PY{n}{y} \PY{o}{==} \PY{l+m+mi}{1}\PY{p}{)} \PY{o}{\PYZam{}} \PY{p}{(}\PY{n}{thresholded\PYZus{}preds} \PY{o}{==} \PY{l+m+mi}{1}\PY{p}{)}\PY{p}{)}
             
             \PY{k}{return} \PY{n}{TP}
         
         \PY{k}{def} \PY{n+nf}{true\PYZus{}negatives}\PY{p}{(}\PY{n}{y}\PY{p}{,} \PY{n}{pred}\PY{p}{,} \PY{n}{th}\PY{o}{=}\PY{l+m+mf}{0.5}\PY{p}{)}\PY{p}{:}
             \PY{l+s+sd}{\PYZdq{}\PYZdq{}\PYZdq{}}
         \PY{l+s+sd}{    Count true negatives.}
         
         \PY{l+s+sd}{    Args:}
         \PY{l+s+sd}{        y (np.array): ground truth, size (n\PYZus{}examples)}
         \PY{l+s+sd}{        pred (np.array): model output, size (n\PYZus{}examples)}
         \PY{l+s+sd}{        th (float): cutoff value for positive prediction from model}
         \PY{l+s+sd}{    Returns:}
         \PY{l+s+sd}{        TN (int): true negatives}
         \PY{l+s+sd}{    \PYZdq{}\PYZdq{}\PYZdq{}}
             \PY{n}{TN} \PY{o}{=} \PY{l+m+mi}{0}
             
             \PY{c+c1}{\PYZsh{} get thresholded predictions}
             \PY{n}{thresholded\PYZus{}preds} \PY{o}{=} \PY{n}{pred} \PY{o}{\PYZgt{}}\PY{o}{=} \PY{n}{th}
         
             \PY{c+c1}{\PYZsh{}\PYZsh{}\PYZsh{} START CODE HERE (REPLACE INSTANCES OF \PYZsq{}None\PYZsq{} with your code) \PYZsh{}\PYZsh{}\PYZsh{}}
             
             \PY{c+c1}{\PYZsh{} compute TN}
             \PY{n}{TN} \PY{o}{=} \PY{n}{np}\PY{o}{.}\PY{n}{sum}\PY{p}{(}\PY{p}{(}\PY{n}{y} \PY{o}{==} \PY{l+m+mi}{0}\PY{p}{)} \PY{o}{\PYZam{}} \PY{p}{(}\PY{n}{thresholded\PYZus{}preds} \PY{o}{==} \PY{l+m+mi}{0}\PY{p}{)}\PY{p}{)} 
             \PY{c+c1}{\PYZsh{}\PYZsh{}\PYZsh{} END CODE HERE \PYZsh{}\PYZsh{}\PYZsh{}}
             
             \PY{k}{return} \PY{n}{TN}
         
         \PY{k}{def} \PY{n+nf}{false\PYZus{}positives}\PY{p}{(}\PY{n}{y}\PY{p}{,} \PY{n}{pred}\PY{p}{,} \PY{n}{th}\PY{o}{=}\PY{l+m+mf}{0.5}\PY{p}{)}\PY{p}{:}
             \PY{l+s+sd}{\PYZdq{}\PYZdq{}\PYZdq{}}
         \PY{l+s+sd}{    Count false positives.}
         
         \PY{l+s+sd}{    Args:}
         \PY{l+s+sd}{        y (np.array): ground truth, size (n\PYZus{}examples)}
         \PY{l+s+sd}{        pred (np.array): model output, size (n\PYZus{}examples)}
         \PY{l+s+sd}{        th (float): cutoff value for positive prediction from model}
         \PY{l+s+sd}{    Returns:}
         \PY{l+s+sd}{        FP (int): false positives}
         \PY{l+s+sd}{    \PYZdq{}\PYZdq{}\PYZdq{}}
             \PY{n}{FP} \PY{o}{=} \PY{l+m+mi}{0}
             
             \PY{c+c1}{\PYZsh{} get thresholded predictions}
             \PY{n}{thresholded\PYZus{}preds} \PY{o}{=} \PY{n}{pred} \PY{o}{\PYZgt{}}\PY{o}{=} \PY{n}{th}
             
             \PY{c+c1}{\PYZsh{}\PYZsh{}\PYZsh{} START CODE HERE (REPLACE INSTANCES OF \PYZsq{}None\PYZsq{} with your code) \PYZsh{}\PYZsh{}\PYZsh{}}
         
             \PY{c+c1}{\PYZsh{} compute FP}
             \PY{n}{FP} \PY{o}{=} \PY{n}{np}\PY{o}{.}\PY{n}{sum}\PY{p}{(}\PY{p}{(}\PY{n}{y} \PY{o}{==} \PY{l+m+mi}{0}\PY{p}{)} \PY{o}{\PYZam{}} \PY{p}{(}\PY{n}{thresholded\PYZus{}preds} \PY{o}{==} \PY{l+m+mi}{1}\PY{p}{)}\PY{p}{)}
             
             \PY{c+c1}{\PYZsh{}\PYZsh{}\PYZsh{} END CODE HERE \PYZsh{}\PYZsh{}\PYZsh{}}
             
             \PY{k}{return} \PY{n}{FP}
         
         \PY{k}{def} \PY{n+nf}{false\PYZus{}negatives}\PY{p}{(}\PY{n}{y}\PY{p}{,} \PY{n}{pred}\PY{p}{,} \PY{n}{th}\PY{o}{=}\PY{l+m+mf}{0.5}\PY{p}{)}\PY{p}{:}
             \PY{l+s+sd}{\PYZdq{}\PYZdq{}\PYZdq{}}
         \PY{l+s+sd}{    Count false positives.}
         
         \PY{l+s+sd}{    Args:}
         \PY{l+s+sd}{        y (np.array): ground truth, size (n\PYZus{}examples)}
         \PY{l+s+sd}{        pred (np.array): model output, size (n\PYZus{}examples)}
         \PY{l+s+sd}{        th (float): cutoff value for positive prediction from model}
         \PY{l+s+sd}{    Returns:}
         \PY{l+s+sd}{        FN (int): false negatives}
         \PY{l+s+sd}{    \PYZdq{}\PYZdq{}\PYZdq{}}
             \PY{n}{FN} \PY{o}{=} \PY{l+m+mi}{0}
             
             \PY{c+c1}{\PYZsh{} get thresholded predictions}
             \PY{n}{thresholded\PYZus{}preds} \PY{o}{=} \PY{n}{pred} \PY{o}{\PYZgt{}} \PY{n}{th}
         
             \PY{c+c1}{\PYZsh{}\PYZsh{}\PYZsh{} START CODE HERE (REPLACE INSTANCES OF \PYZsq{}None\PYZsq{} with your code) \PYZsh{}\PYZsh{}\PYZsh{}}
             
             \PY{c+c1}{\PYZsh{} compute FN}
             \PY{n}{FN} \PY{o}{=} \PY{n}{np}\PY{o}{.}\PY{n}{sum}\PY{p}{(}\PY{p}{(}\PY{n}{y} \PY{o}{==} \PY{l+m+mi}{1}\PY{p}{)} \PY{o}{\PYZam{}} \PY{p}{(}\PY{n}{thresholded\PYZus{}preds} \PY{o}{==} \PY{l+m+mi}{0}\PY{p}{)}\PY{p}{)}
             \PY{c+c1}{\PYZsh{}\PYZsh{}\PYZsh{} END CODE HERE \PYZsh{}\PYZsh{}\PYZsh{}}
             
             \PY{k}{return} \PY{n}{FN}
\end{Verbatim}


    \begin{Verbatim}[commandchars=\\\{\}]
{\color{incolor}In [{\color{incolor}11}]:} \PY{c+c1}{\PYZsh{}\PYZsh{}\PYZsh{} do not modify this cell    }
         \PY{n}{get\PYZus{}tp\PYZus{}tn\PYZus{}fp\PYZus{}fn\PYZus{}test}\PY{p}{(}\PY{n}{true\PYZus{}positives}\PY{p}{,} \PY{n}{true\PYZus{}negatives}\PY{p}{,} \PY{n}{false\PYZus{}positives}\PY{p}{,} \PY{n}{false\PYZus{}negatives}\PY{p}{)}    
\end{Verbatim}


    
    \begin{verbatim}
    y_test  preds_test category
0        1         0.8       TP
1        1         0.7       TP
2        0         0.4       TN
3        0         0.3       TN
4        0         0.2       TN
5        0         0.5       FP
6        0         0.6       FP
7        0         0.7       FP
8        0         0.8       FP
9        1         0.1       FN
10       1         0.2       FN
11       1         0.3       FN
12       1         0.4       FN
13       1         0.0       FN
    \end{verbatim}

    
    \begin{Verbatim}[commandchars=\\\{\}]
Your functions calcualted: 
    TP: 2
    TN: 3
    FP: 4
    FN: 4
    
\textcolor{ansi-green-intense}{ All tests passed.
}\textcolor{ansi-green-intense}{ All tests passed.
}\textcolor{ansi-green-intense}{ All tests passed.
Error: Wrong output in false\_negatives
}\textcolor{ansi-green-intense}{ 2  Tests passed
}\textcolor{ansi-red-intense}{ 1  Tests failed
}
    \end{Verbatim}

    \begin{Verbatim}[commandchars=\\\{\}]

        ---------------------------------------------------------------------------

        AssertionError                            Traceback (most recent call last)

        <ipython-input-11-7b983657bb71> in <module>()
          1 \#\#\# do not modify this cell
    ----> 2 get\_tp\_tn\_fp\_fn\_test(true\_positives, true\_negatives, false\_positives, false\_negatives)
    

        \textasciitilde{}/work/W2A1/public\_tests.py in get\_tp\_tn\_fp\_fn\_test(target\_1, target\_2, target\_3, target\_4)
        119     ]
        120 
    --> 121     multiple\_test(test\_cases\_4, target\_4)
        122 
        123 \#\#\# ex2


        \textasciitilde{}/work/W2A1/test\_utils.py in multiple\_test(test\_cases, target)
        119         print('\textbackslash{}033[92m', success," Tests passed")
        120         print('\textbackslash{}033[91m', len(test\_cases) - success, " Tests failed")
    --> 121         raise AssertionError("Not all tests were passed for \{\}. Check your equations and avoid using global variables inside the function.".format(target.\_\_name\_\_))
        122 
        123 


        AssertionError: Not all tests were passed for false\_negatives. Check your equations and avoid using global variables inside the function.

    \end{Verbatim}

    \hypertarget{expected-output}{%
\subparagraph{Expected output}\label{expected-output}}

\begin{Shaded}
\begin{Highlighting}[]
\NormalTok{Your functions calcualted: }
\NormalTok{    TP: }\DecValTok{2}
\NormalTok{    TN: }\DecValTok{3}
\NormalTok{    FP: }\DecValTok{4}
\NormalTok{    FN: }\DecValTok{5}
\end{Highlighting}
\end{Shaded}

\begin{verbatim}
 All tests passed.
 All tests passed.
 All tests passed.
 All tests passed.
\end{verbatim}

    Run the next cell to see a summary of evaluative metrics for the model
predictions for each class.

    \begin{Verbatim}[commandchars=\\\{\}]
{\color{incolor}In [{\color{incolor}16}]:} \PY{n}{util}\PY{o}{.}\PY{n}{get\PYZus{}performance\PYZus{}metrics}\PY{p}{(}\PY{n}{y}\PY{p}{,} \PY{n}{pred}\PY{p}{,} \PY{n}{class\PYZus{}labels}\PY{p}{)}
\end{Verbatim}


\begin{Verbatim}[commandchars=\\\{\}]
{\color{outcolor}Out[{\color{outcolor}16}]:}                      TP   TN   FP  FN     Accuracy   Prevalence  Sensitivity  \textbackslash{}
                                                                                        
         Cardiomegaly         16  814  169   1  Not Defined  Not Defined  Not Defined   
         Emphysema            20  869  103   8  Not Defined  Not Defined  Not Defined   
         Effusion             99  690  196  15  Not Defined  Not Defined  Not Defined   
         Hernia                1  743  255   1  Not Defined  Not Defined  Not Defined   
         Infiltration        114  543  265  78  Not Defined  Not Defined  Not Defined   
         Mass                 40  789  158  13  Not Defined  Not Defined  Not Defined   
         Nodule               28  731  220  21  Not Defined  Not Defined  Not Defined   
         Atelectasis          64  657  249  30  Not Defined  Not Defined  Not Defined   
         Pneumothorax         24  785  183   8  Not Defined  Not Defined  Not Defined   
         Pleural\_Thickening   24  713  259   4  Not Defined  Not Defined  Not Defined   
         Pneumonia            14  661  320   5  Not Defined  Not Defined  Not Defined   
         Fibrosis             10  725  261   4  Not Defined  Not Defined  Not Defined   
         Edema                15  767  213   5  Not Defined  Not Defined  Not Defined   
         Consolidation        36  658  297   9  Not Defined  Not Defined  Not Defined   
         
                             Specificity          PPV          NPV          AUC  \textbackslash{}
                                                                                  
         Cardiomegaly        Not Defined  Not Defined  Not Defined  Not Defined   
         Emphysema           Not Defined  Not Defined  Not Defined  Not Defined   
         Effusion            Not Defined  Not Defined  Not Defined  Not Defined   
         Hernia              Not Defined  Not Defined  Not Defined  Not Defined   
         Infiltration        Not Defined  Not Defined  Not Defined  Not Defined   
         Mass                Not Defined  Not Defined  Not Defined  Not Defined   
         Nodule              Not Defined  Not Defined  Not Defined  Not Defined   
         Atelectasis         Not Defined  Not Defined  Not Defined  Not Defined   
         Pneumothorax        Not Defined  Not Defined  Not Defined  Not Defined   
         Pleural\_Thickening  Not Defined  Not Defined  Not Defined  Not Defined   
         Pneumonia           Not Defined  Not Defined  Not Defined  Not Defined   
         Fibrosis            Not Defined  Not Defined  Not Defined  Not Defined   
         Edema               Not Defined  Not Defined  Not Defined  Not Defined   
         Consolidation       Not Defined  Not Defined  Not Defined  Not Defined   
         
                                      F1 Threshold  
                                                    
         Cardiomegaly        Not Defined       0.5  
         Emphysema           Not Defined       0.5  
         Effusion            Not Defined       0.5  
         Hernia              Not Defined       0.5  
         Infiltration        Not Defined       0.5  
         Mass                Not Defined       0.5  
         Nodule              Not Defined       0.5  
         Atelectasis         Not Defined       0.5  
         Pneumothorax        Not Defined       0.5  
         Pleural\_Thickening  Not Defined       0.5  
         Pneumonia           Not Defined       0.5  
         Fibrosis            Not Defined       0.5  
         Edema               Not Defined       0.5  
         Consolidation       Not Defined       0.5  
\end{Verbatim}
            
    Right now it only has TP, TN, FP, FN. Throughout this assignment we'll
fill in all the other metrics to learn more about our model performance.

     \#\#\# 3.2 - Accuracy

Let's use a threshold of .5 for the probability cutoff for our
predictions for all classes and calculate our model's accuracy as we
would normally do in a machine learning problem.

\[accuracy = \frac{\text{true positives} + \text{true negatives}}{\text{true positives} + \text{true negatives} + \text{false positives} + \text{false negatives}}\]

 \#\#\# Exercise 2 - get\_accuracy

Use this formula to compute accuracy below:

     Hints

Remember to set the value for the threshold when calling the functions.

    \begin{Verbatim}[commandchars=\\\{\}]
{\color{incolor}In [{\color{incolor}17}]:} \PY{c+c1}{\PYZsh{} UNQ\PYZus{}C2 (UNIQUE CELL IDENTIFIER, DO NOT EDIT)}
         \PY{k}{def} \PY{n+nf}{get\PYZus{}accuracy}\PY{p}{(}\PY{n}{y}\PY{p}{,} \PY{n}{pred}\PY{p}{,} \PY{n}{th}\PY{o}{=}\PY{l+m+mf}{0.5}\PY{p}{)}\PY{p}{:}
             \PY{l+s+sd}{\PYZdq{}\PYZdq{}\PYZdq{}}
         \PY{l+s+sd}{    Compute accuracy of predictions at threshold.}
         
         \PY{l+s+sd}{    Args:}
         \PY{l+s+sd}{        y (np.array): ground truth, size (n\PYZus{}examples)}
         \PY{l+s+sd}{        pred (np.array): model output, size (n\PYZus{}examples)}
         \PY{l+s+sd}{        th (float): cutoff value for positive prediction from model}
         \PY{l+s+sd}{    Returns:}
         \PY{l+s+sd}{        accuracy (float): accuracy of predictions at threshold}
         \PY{l+s+sd}{    \PYZdq{}\PYZdq{}\PYZdq{}}
             \PY{n}{accuracy} \PY{o}{=} \PY{l+m+mf}{0.0}
             
             \PY{c+c1}{\PYZsh{}\PYZsh{}\PYZsh{} START CODE HERE (REPLACE INSTANCES OF \PYZsq{}None\PYZsq{} with your code) \PYZsh{}\PYZsh{}\PYZsh{}}
             
             \PY{c+c1}{\PYZsh{} get TP, FP, TN, FN using our previously defined functions}
             \PY{n}{TP} \PY{o}{=} \PY{n}{true\PYZus{}positives}\PY{p}{(}\PY{n}{y}\PY{p}{,} \PY{n}{pred}\PY{p}{,} \PY{n}{th}\PY{o}{=}\PY{n}{th}\PY{p}{)}
             \PY{n}{FP} \PY{o}{=} \PY{n}{false\PYZus{}positives}\PY{p}{(}\PY{n}{y}\PY{p}{,} \PY{n}{pred}\PY{p}{,} \PY{n}{th}\PY{o}{=}\PY{n}{th}\PY{p}{)}
             \PY{n}{TN} \PY{o}{=} \PY{n}{true\PYZus{}negatives}\PY{p}{(}\PY{n}{y}\PY{p}{,} \PY{n}{pred}\PY{p}{,} \PY{n}{th}\PY{o}{=}\PY{n}{th}\PY{p}{)}
             \PY{n}{FN} \PY{o}{=} \PY{n}{false\PYZus{}negatives}\PY{p}{(}\PY{n}{y}\PY{p}{,} \PY{n}{pred}\PY{p}{,} \PY{n}{th}\PY{o}{=}\PY{n}{th}\PY{p}{)}
         
             \PY{c+c1}{\PYZsh{} Compute accuracy using TP, FP, TN, FN}
             \PY{n}{accuracy} \PY{o}{=} \PY{p}{(}\PY{n}{TP} \PY{o}{+} \PY{n}{TN}\PY{p}{)} \PY{o}{/} \PY{p}{(}\PY{n}{TP} \PY{o}{+} \PY{n}{FP} \PY{o}{+} \PY{n}{FN} \PY{o}{+} \PY{n}{TN}\PY{p}{)}
             
             \PY{c+c1}{\PYZsh{}\PYZsh{}\PYZsh{} END CODE HERE \PYZsh{}\PYZsh{}\PYZsh{}}
             
             \PY{k}{return} \PY{n}{accuracy}
\end{Verbatim}


    \begin{Verbatim}[commandchars=\\\{\}]
{\color{incolor}In [{\color{incolor}18}]:} \PY{c+c1}{\PYZsh{}\PYZsh{}\PYZsh{} do not modify this cell    }
         \PY{n}{get\PYZus{}accuracy\PYZus{}test}\PY{p}{(}\PY{n}{get\PYZus{}accuracy}\PY{p}{)}
\end{Verbatim}


    \begin{Verbatim}[commandchars=\\\{\}]
Test Case:

Test Labels:	   [1 0 0 1 1]
Test Predictions:  [0.8 0.8 0.4 0.6 0.3]
Threshold:	   0.5
Computed Accuracy: 0.6 

\textcolor{ansi-green-intense}{ All tests passed.
}
    \end{Verbatim}

    \hypertarget{expected-output}{%
\paragraph{Expected output:}\label{expected-output}}

\begin{Shaded}
\begin{Highlighting}[]
\NormalTok{Test Case:}

\NormalTok{Test Labels:       [}\DecValTok{1} \DecValTok{0} \DecValTok{0} \DecValTok{1} \DecValTok{1}\NormalTok{]}
\NormalTok{Test Predictions:  [}\FloatTok{0.8} \FloatTok{0.8} \FloatTok{0.4} \FloatTok{0.6} \FloatTok{0.3}\NormalTok{]}
\NormalTok{Threshold:       }\FloatTok{0.5}
\NormalTok{Computed Accuracy: }\FloatTok{0.6} 
\end{Highlighting}
\end{Shaded}

\begin{verbatim}
 All tests passed.
\end{verbatim}

    Run the next cell to see the accuracy of the model output for each
class, as well as the number of true positives, true negatives, false
positives, and false negatives.

    \begin{Verbatim}[commandchars=\\\{\}]
{\color{incolor}In [{\color{incolor} }]:} \PY{n}{util}\PY{o}{.}\PY{n}{get\PYZus{}performance\PYZus{}metrics}\PY{p}{(}\PY{n}{y}\PY{p}{,} \PY{n}{pred}\PY{p}{,} \PY{n}{class\PYZus{}labels}\PY{p}{,} \PY{n}{acc}\PY{o}{=}\PY{n}{get\PYZus{}accuracy}\PY{p}{)}
\end{Verbatim}


    If we were to judge our model's performance based on the accuracy
metric, we would say that our model is not very accurate for detecting
the \texttt{Infiltration} cases (accuracy of 0.657) but pretty accurate
for detecting \texttt{Emphysema} (accuracy of 0.889).

\textbf{But is that really the case?\ldots{}}

Let's imagine a model that simply predicts that any patient does
\textbf{Not} have \texttt{Emphysema}, regardless of patient's
measurements. Let's calculate the accuracy for such a model.

    \begin{Verbatim}[commandchars=\\\{\}]
{\color{incolor}In [{\color{incolor} }]:} \PY{n}{get\PYZus{}accuracy}\PY{p}{(}\PY{n}{valid\PYZus{}results}\PY{p}{[}\PY{l+s+s2}{\PYZdq{}}\PY{l+s+s2}{Emphysema}\PY{l+s+s2}{\PYZdq{}}\PY{p}{]}\PY{o}{.}\PY{n}{values}\PY{p}{,} \PY{n}{np}\PY{o}{.}\PY{n}{zeros}\PY{p}{(}\PY{n+nb}{len}\PY{p}{(}\PY{n}{valid\PYZus{}results}\PY{p}{)}\PY{p}{)}\PY{p}{)}
\end{Verbatim}


    As you can see above, such a model would be 97\% accurate! Even better
than our deep learning based model.

But is this really a good model? Wouldn't this model be wrong 100\% of
the time if the patient actually had this condition?

In the following sections, we will address this concern with more
advanced model measures - \textbf{sensitivity and specificity} - that
evaluate how well the model predicts positives for patients with the
condition and negatives for cases that actually do not have the
condition.

     \#\#\# 3.3 - Prevalence

Another important concept is \textbf{prevalence}. * In a medical
context, prevalence is the proportion of people in the population who
have the disease (or condition, etc). * In machine learning terms, this
is the proportion of positive examples. The expression for prevalence
is:

\[prevalence = \frac{1}{N} \sum_{i} y_i\]

where \(y_i = 1\) when the example is `positive' (has the disease).

 \#\#\# Exercise 3 - get\_prevalence

Let's measure prevalence for each disease:

     Hints

You can use np.mean to calculate the formula.

Actually, the automatic grader is expecting numpy.mean, so please use it
instead of using an equally valid but different way of calculating the
prevalence. =)

    \begin{Verbatim}[commandchars=\\\{\}]
{\color{incolor}In [{\color{incolor}19}]:} \PY{c+c1}{\PYZsh{} UNQ\PYZus{}C3 (UNIQUE CELL IDENTIFIER, DO NOT EDIT)}
         \PY{k}{def} \PY{n+nf}{get\PYZus{}prevalence}\PY{p}{(}\PY{n}{y}\PY{p}{)}\PY{p}{:}
             \PY{l+s+sd}{\PYZdq{}\PYZdq{}\PYZdq{}}
         \PY{l+s+sd}{    Compute prevalence.}
         
         \PY{l+s+sd}{    Args:}
         \PY{l+s+sd}{        y (np.array): ground truth, size (n\PYZus{}examples)}
         \PY{l+s+sd}{    Returns:}
         \PY{l+s+sd}{        prevalence (float): prevalence of positive cases}
         \PY{l+s+sd}{    \PYZdq{}\PYZdq{}\PYZdq{}}
             \PY{n}{prevalence} \PY{o}{=} \PY{l+m+mf}{0.0}
             
             \PY{c+c1}{\PYZsh{}\PYZsh{}\PYZsh{} START CODE HERE (REPLACE INSTANCES OF \PYZsq{}None\PYZsq{} with your code) \PYZsh{}\PYZsh{}\PYZsh{}}
             
             \PY{n}{prevalence} \PY{o}{=} \PY{n}{np}\PY{o}{.}\PY{n}{mean}\PY{p}{(}\PY{n}{y}\PY{p}{)}
             
             \PY{c+c1}{\PYZsh{}\PYZsh{}\PYZsh{} END CODE HERE \PYZsh{}\PYZsh{}\PYZsh{}}
             
             \PY{k}{return} \PY{n}{prevalence}
\end{Verbatim}


    \begin{Verbatim}[commandchars=\\\{\}]
{\color{incolor}In [{\color{incolor}20}]:} \PY{c+c1}{\PYZsh{}\PYZsh{}\PYZsh{} do npt modify this cell    }
         \PY{n}{get\PYZus{}prevalence\PYZus{}test}\PY{p}{(}\PY{n}{get\PYZus{}prevalence}\PY{p}{)}
\end{Verbatim}


    \begin{Verbatim}[commandchars=\\\{\}]
Test Case:

Test Labels:	      [1 0 0 1 1 0 0 0 0 1]
Computed Prevalence:  0.4 

\textcolor{ansi-green-intense}{ All tests passed.
}
    \end{Verbatim}

    \hypertarget{expected-output}{%
\paragraph{Expected output:}\label{expected-output}}

\begin{Shaded}
\begin{Highlighting}[]
\NormalTok{Test Case:}

\NormalTok{Test Labels:          [}\DecValTok{1} \DecValTok{0} \DecValTok{0} \DecValTok{1} \DecValTok{1} \DecValTok{0} \DecValTok{0} \DecValTok{0} \DecValTok{0} \DecValTok{1}\NormalTok{]}
\NormalTok{Computed Prevalence:  }\FloatTok{0.4}  
\end{Highlighting}
\end{Shaded}

\begin{verbatim}
 All tests passed.
\end{verbatim}

    \begin{Verbatim}[commandchars=\\\{\}]
{\color{incolor}In [{\color{incolor} }]:} \PY{n}{util}\PY{o}{.}\PY{n}{get\PYZus{}performance\PYZus{}metrics}\PY{p}{(}\PY{n}{y}\PY{p}{,} \PY{n}{pred}\PY{p}{,} \PY{n}{class\PYZus{}labels}\PY{p}{,} \PY{n}{acc}\PY{o}{=}\PY{n}{get\PYZus{}accuracy}\PY{p}{,} \PY{n}{prevalence}\PY{o}{=}\PY{n}{get\PYZus{}prevalence}\PY{p}{)}
\end{Verbatim}


    \texttt{Hernia} has a prevalence 0.002, which is the rarest among the
studied conditions in our dataset.

     \#\#\# 3.4 Sensitivity and Specificity

Sensitivity and specificity are two of the most prominent numbers that
are used to measure diagnostics tests. - Sensitivity is the probability
that our test outputs positive given that the case is actually positive.
- Specificity is the probability that the test outputs negative given
that the case is actually negative.

We can phrase this easily in terms of true positives, true negatives,
false positives, and false negatives:

\[sensitivity = \frac{\text{true positives}}{\text{true positives} + \text{false negatives}}\]

\[specificity = \frac{\text{true negatives}}{\text{true negatives} + \text{false positives}}\]

 \#\#\# Exercise 4 - get\_sensitivity and get\_specificity

Let's calculate sensitivity and specificity for our model:

    \begin{Verbatim}[commandchars=\\\{\}]
{\color{incolor}In [{\color{incolor}26}]:} \PY{c+c1}{\PYZsh{} UNQ\PYZus{}C4 (UNIQUE CELL IDENTIFIER, DO NOT EDIT)}
         \PY{k}{def} \PY{n+nf}{get\PYZus{}sensitivity}\PY{p}{(}\PY{n}{y}\PY{p}{,} \PY{n}{pred}\PY{p}{,} \PY{n}{th}\PY{o}{=}\PY{l+m+mf}{0.5}\PY{p}{)}\PY{p}{:}
             \PY{l+s+sd}{\PYZdq{}\PYZdq{}\PYZdq{}}
         \PY{l+s+sd}{    Compute sensitivity of predictions at threshold.}
         
         \PY{l+s+sd}{    Args:}
         \PY{l+s+sd}{        y (np.array): ground truth, size (n\PYZus{}examples)}
         \PY{l+s+sd}{        pred (np.array): model output, size (n\PYZus{}examples)}
         \PY{l+s+sd}{        th (float): cutoff value for positive prediction from model}
         \PY{l+s+sd}{    Returns:}
         \PY{l+s+sd}{        sensitivity (float): probability that our test outputs positive given that the case is actually positive}
         \PY{l+s+sd}{    \PYZdq{}\PYZdq{}\PYZdq{}}
             \PY{n}{sensitivity} \PY{o}{=} \PY{l+m+mf}{0.0}
             
             \PY{c+c1}{\PYZsh{}\PYZsh{}\PYZsh{} START CODE HERE (REPLACE INSTANCES OF \PYZsq{}None\PYZsq{} with your code) \PYZsh{}\PYZsh{}\PYZsh{}}
             
             \PY{c+c1}{\PYZsh{} get TP and FN using our previously defined functions}
             \PY{n}{TP} \PY{o}{=} \PY{n}{true\PYZus{}positives}\PY{p}{(}\PY{n}{y}\PY{p}{,} \PY{n}{pred}\PY{p}{,} \PY{n}{th}\PY{p}{)}
             \PY{n}{FN} \PY{o}{=} \PY{n}{false\PYZus{}negatives}\PY{p}{(}\PY{n}{y}\PY{p}{,} \PY{n}{pred}\PY{p}{,} \PY{n}{th}\PY{p}{)}
         
             \PY{c+c1}{\PYZsh{} use TP and FN to compute sensitivity}
             \PY{n}{sensitivity} \PY{o}{=} \PY{n}{TP} \PY{o}{/} \PY{p}{(}\PY{n}{TP} \PY{o}{+} \PY{n}{FN}\PY{p}{)}
             
             \PY{c+c1}{\PYZsh{}\PYZsh{}\PYZsh{} END CODE HERE \PYZsh{}\PYZsh{}\PYZsh{}}
             
             \PY{k}{return} \PY{n}{sensitivity}
         
         \PY{k}{def} \PY{n+nf}{get\PYZus{}specificity}\PY{p}{(}\PY{n}{y}\PY{p}{,} \PY{n}{pred}\PY{p}{,} \PY{n}{th}\PY{o}{=}\PY{l+m+mf}{0.5}\PY{p}{)}\PY{p}{:}
             \PY{l+s+sd}{\PYZdq{}\PYZdq{}\PYZdq{}}
         \PY{l+s+sd}{    Compute specificity of predictions at threshold.}
         
         \PY{l+s+sd}{    Args:}
         \PY{l+s+sd}{        y (np.array): ground truth, size (n\PYZus{}examples)}
         \PY{l+s+sd}{        pred (np.array): model output, size (n\PYZus{}examples)}
         \PY{l+s+sd}{        th (float): cutoff value for positive prediction from model}
         \PY{l+s+sd}{    Returns:}
         \PY{l+s+sd}{        specificity (float): probability that the test outputs negative given that the case is actually negative}
         \PY{l+s+sd}{    \PYZdq{}\PYZdq{}\PYZdq{}}
             \PY{n}{specificity} \PY{o}{=} \PY{l+m+mf}{0.0}
             
             \PY{c+c1}{\PYZsh{}\PYZsh{}\PYZsh{} START CODE HERE (REPLACE INSTANCES OF \PYZsq{}None\PYZsq{} with your code) \PYZsh{}\PYZsh{}\PYZsh{}}
             
             \PY{c+c1}{\PYZsh{} get TN and FP using our previously defined functions}
             \PY{n}{TN} \PY{o}{=} \PY{n}{true\PYZus{}negatives}\PY{p}{(}\PY{n}{y}\PY{p}{,} \PY{n}{pred}\PY{p}{,} \PY{n}{th}\PY{p}{)}
             \PY{n}{FP} \PY{o}{=} \PY{n}{false\PYZus{}positives}\PY{p}{(}\PY{n}{y}\PY{p}{,} \PY{n}{pred}\PY{p}{,} \PY{n}{th}\PY{p}{)}
             
             \PY{c+c1}{\PYZsh{} use TN and FP to compute specificity }
             \PY{n}{specificity} \PY{o}{=} \PY{n}{TN} \PY{o}{/} \PY{p}{(}\PY{n}{TN} \PY{o}{+} \PY{n}{FP}\PY{p}{)}
         
             
             \PY{c+c1}{\PYZsh{}\PYZsh{}\PYZsh{} END CODE HERE \PYZsh{}\PYZsh{}\PYZsh{}}
             
             \PY{k}{return} \PY{n}{specificity}
\end{Verbatim}


    \begin{Verbatim}[commandchars=\\\{\}]
{\color{incolor}In [{\color{incolor}27}]:} \PY{c+c1}{\PYZsh{}\PYZsh{}\PYZsh{} do not modify this cell    }
         \PY{n}{get\PYZus{}sensitivity\PYZus{}specificity\PYZus{}test}\PY{p}{(}\PY{n}{get\PYZus{}sensitivity}\PY{p}{,} \PY{n}{get\PYZus{}specificity}\PY{p}{)}
\end{Verbatim}


    \begin{Verbatim}[commandchars=\\\{\}]
Test Case:

Test Labels:	       [1 0 0 1 1]
Test Predictions:      [1 0 0 1 1]
Threshold:	       0.5
Computed Sensitivity:  0.6666666666666666
Computed Specificity:  0.5 

\textcolor{ansi-green-intense}{ All tests passed.
}\textcolor{ansi-green-intense}{ All tests passed.
}
    \end{Verbatim}

    \hypertarget{expected-output}{%
\paragraph{Expected output:}\label{expected-output}}

\begin{Shaded}
\begin{Highlighting}[]
\NormalTok{Test Case:}

\NormalTok{Test Labels:           [}\DecValTok{1} \DecValTok{0} \DecValTok{0} \DecValTok{1} \DecValTok{1}\NormalTok{]}
\NormalTok{Test Predictions:      [}\DecValTok{1} \DecValTok{0} \DecValTok{0} \DecValTok{1} \DecValTok{1}\NormalTok{]}
\NormalTok{Threshold:           }\FloatTok{0.5}
\NormalTok{Computed Sensitivity:  }\FloatTok{0.6666666666666666}
\NormalTok{Computed Specificity:  }\FloatTok{0.5} 
\end{Highlighting}
\end{Shaded}

\begin{verbatim}
 All tests passed.
 All tests passed.
\end{verbatim}

    \begin{Verbatim}[commandchars=\\\{\}]
{\color{incolor}In [{\color{incolor}28}]:} \PY{n}{util}\PY{o}{.}\PY{n}{get\PYZus{}performance\PYZus{}metrics}\PY{p}{(}\PY{n}{y}\PY{p}{,} \PY{n}{pred}\PY{p}{,} \PY{n}{class\PYZus{}labels}\PY{p}{,} \PY{n}{acc}\PY{o}{=}\PY{n}{get\PYZus{}accuracy}\PY{p}{,} \PY{n}{prevalence}\PY{o}{=}\PY{n}{get\PYZus{}prevalence}\PY{p}{,} 
                                 \PY{n}{sens}\PY{o}{=}\PY{n}{get\PYZus{}sensitivity}\PY{p}{,} \PY{n}{spec}\PY{o}{=}\PY{n}{get\PYZus{}specificity}\PY{p}{)}
\end{Verbatim}


\begin{Verbatim}[commandchars=\\\{\}]
{\color{outcolor}Out[{\color{outcolor}28}]:}                      TP   TN   FP  FN Accuracy Prevalence Sensitivity  \textbackslash{}
                                                                                 
         Cardiomegaly         16  814  169   1     0.83      0.017       0.941   
         Emphysema            20  869  103   8    0.889      0.028       0.714   
         Effusion             99  690  196  15    0.789      0.114       0.868   
         Hernia                1  743  255   1    0.744      0.002         0.5   
         Infiltration        114  543  265  78    0.657      0.192       0.594   
         Mass                 40  789  158  13    0.829      0.053       0.755   
         Nodule               28  731  220  21    0.759      0.049       0.571   
         Atelectasis          64  657  249  30    0.721      0.094       0.681   
         Pneumothorax         24  785  183   8    0.809      0.032        0.75   
         Pleural\_Thickening   24  713  259   4    0.737      0.028       0.857   
         Pneumonia            14  661  320   5    0.675      0.019       0.737   
         Fibrosis             10  725  261   4    0.735      0.014       0.714   
         Edema                15  767  213   5    0.782       0.02        0.75   
         Consolidation        36  658  297   9    0.694      0.045         0.8   
         
                            Specificity          PPV          NPV          AUC  \textbackslash{}
                                                                                 
         Cardiomegaly             0.828  Not Defined  Not Defined  Not Defined   
         Emphysema                0.894  Not Defined  Not Defined  Not Defined   
         Effusion                 0.779  Not Defined  Not Defined  Not Defined   
         Hernia                   0.744  Not Defined  Not Defined  Not Defined   
         Infiltration             0.672  Not Defined  Not Defined  Not Defined   
         Mass                     0.833  Not Defined  Not Defined  Not Defined   
         Nodule                   0.769  Not Defined  Not Defined  Not Defined   
         Atelectasis              0.725  Not Defined  Not Defined  Not Defined   
         Pneumothorax             0.811  Not Defined  Not Defined  Not Defined   
         Pleural\_Thickening       0.734  Not Defined  Not Defined  Not Defined   
         Pneumonia                0.674  Not Defined  Not Defined  Not Defined   
         Fibrosis                 0.735  Not Defined  Not Defined  Not Defined   
         Edema                    0.783  Not Defined  Not Defined  Not Defined   
         Consolidation            0.689  Not Defined  Not Defined  Not Defined   
         
                                      F1 Threshold  
                                                    
         Cardiomegaly        Not Defined       0.5  
         Emphysema           Not Defined       0.5  
         Effusion            Not Defined       0.5  
         Hernia              Not Defined       0.5  
         Infiltration        Not Defined       0.5  
         Mass                Not Defined       0.5  
         Nodule              Not Defined       0.5  
         Atelectasis         Not Defined       0.5  
         Pneumothorax        Not Defined       0.5  
         Pleural\_Thickening  Not Defined       0.5  
         Pneumonia           Not Defined       0.5  
         Fibrosis            Not Defined       0.5  
         Edema               Not Defined       0.5  
         Consolidation       Not Defined       0.5  
\end{Verbatim}
            
    Note that specificity and sensitivity do not depend on the prevalence of
the positive class in the dataset. * This is because the statistics are
only computed within people of the same class * Sensitivity only
considers output on people in the positive class * Similarly,
specificity only considers output on people in the negative class.

     \#\#\# 3.5 PPV and NPV

Diagnostically, however, sensitivity and specificity are not helpful.
Sensitivity, for example, tells us the probability our test outputs
positive given that the person already has the condition. Here, we are
conditioning on the thing we would like to find out (whether the patient
has the condition)!

What would be more helpful is the probability that the person has the
disease given that our test outputs positive. That brings us to positive
predictive value (PPV) and negative predictive value (NPV).

\begin{itemize}
\tightlist
\item
  Positive predictive value (PPV) is the probability that subjects with
  a positive screening test truly have the disease.
\item
  Negative predictive value (NPV) is the probability that subjects with
  a negative screening test truly don't have the disease.
\end{itemize}

Again, we can formulate these in terms of true positives, true
negatives, false positives, and false negatives:

\[PPV = \frac{\text{true positives}}{\text{true positives} + \text{false positives}}\]

\[NPV = \frac{\text{true negatives}}{\text{true negatives} + \text{false negatives}}\]

 \#\#\# Exercise 5 - get\_ppv and get\_npv

Let's calculate PPV \& NPV for our model:

    \begin{Verbatim}[commandchars=\\\{\}]
{\color{incolor}In [{\color{incolor}30}]:} \PY{c+c1}{\PYZsh{} UNQ\PYZus{}C5 (UNIQUE CELL IDENTIFIER, DO NOT EDIT)}
         \PY{k}{def} \PY{n+nf}{get\PYZus{}ppv}\PY{p}{(}\PY{n}{y}\PY{p}{,} \PY{n}{pred}\PY{p}{,} \PY{n}{th}\PY{o}{=}\PY{l+m+mf}{0.5}\PY{p}{)}\PY{p}{:}
             \PY{l+s+sd}{\PYZdq{}\PYZdq{}\PYZdq{}}
         \PY{l+s+sd}{    Compute PPV of predictions at threshold.}
         
         \PY{l+s+sd}{    Args:}
         \PY{l+s+sd}{        y (np.array): ground truth, size (n\PYZus{}examples)}
         \PY{l+s+sd}{        pred (np.array): model output, size (n\PYZus{}examples)}
         \PY{l+s+sd}{        th (float): cutoff value for positive prediction from model}
         \PY{l+s+sd}{    Returns:}
         \PY{l+s+sd}{        PPV (float): positive predictive value of predictions at threshold}
         \PY{l+s+sd}{    \PYZdq{}\PYZdq{}\PYZdq{}}
             \PY{n}{PPV} \PY{o}{=} \PY{l+m+mf}{0.0}
             
             \PY{c+c1}{\PYZsh{}\PYZsh{}\PYZsh{} START CODE HERE (REPLACE INSTANCES OF \PYZsq{}None\PYZsq{} with your code) \PYZsh{}\PYZsh{}\PYZsh{}}
             
             \PY{c+c1}{\PYZsh{} get TP and FP using our previously defined functions}
             \PY{n}{TP} \PY{o}{=} \PY{n}{true\PYZus{}positives}\PY{p}{(}\PY{n}{y}\PY{p}{,} \PY{n}{pred}\PY{p}{,} \PY{n}{th}\PY{p}{)}
             \PY{n}{FP} \PY{o}{=} \PY{n}{false\PYZus{}positives}\PY{p}{(}\PY{n}{y}\PY{p}{,} \PY{n}{pred}\PY{p}{,} \PY{n}{th}\PY{p}{)}
             \PY{c+c1}{\PYZsh{} use TP and FP to compute PPV}
             \PY{n}{PPV} \PY{o}{=} \PY{n}{TP} \PY{o}{/} \PY{p}{(}\PY{n}{TP} \PY{o}{+} \PY{n}{FP}\PY{p}{)}
         
             
             \PY{c+c1}{\PYZsh{}\PYZsh{}\PYZsh{} END CODE HERE \PYZsh{}\PYZsh{}\PYZsh{}}
             
             \PY{k}{return} \PY{n}{PPV}
         
         \PY{k}{def} \PY{n+nf}{get\PYZus{}npv}\PY{p}{(}\PY{n}{y}\PY{p}{,} \PY{n}{pred}\PY{p}{,} \PY{n}{th}\PY{o}{=}\PY{l+m+mf}{0.5}\PY{p}{)}\PY{p}{:}
             \PY{l+s+sd}{\PYZdq{}\PYZdq{}\PYZdq{}}
         \PY{l+s+sd}{    Compute NPV of predictions at threshold.}
         
         \PY{l+s+sd}{    Args:}
         \PY{l+s+sd}{        y (np.array): ground truth, size (n\PYZus{}examples)}
         \PY{l+s+sd}{        pred (np.array): model output, size (n\PYZus{}examples)}
         \PY{l+s+sd}{        th (float): cutoff value for positive prediction from model}
         \PY{l+s+sd}{    Returns:}
         \PY{l+s+sd}{        NPV (float): negative predictive value of predictions at threshold}
         \PY{l+s+sd}{    \PYZdq{}\PYZdq{}\PYZdq{}}
             \PY{n}{NPV} \PY{o}{=} \PY{l+m+mf}{0.0}
             
             \PY{c+c1}{\PYZsh{}\PYZsh{}\PYZsh{} START CODE HERE (REPLACE INSTANCES OF \PYZsq{}None\PYZsq{} with your code) \PYZsh{}\PYZsh{}\PYZsh{}}
             
             \PY{c+c1}{\PYZsh{} get TN and FN using our previously defined functions}
             \PY{n}{TN} \PY{o}{=} \PY{n}{true\PYZus{}negatives}\PY{p}{(}\PY{n}{y}\PY{p}{,} \PY{n}{pred}\PY{p}{,} \PY{n}{th}\PY{p}{)}
             \PY{n}{FN} \PY{o}{=} \PY{n}{false\PYZus{}negatives}\PY{p}{(}\PY{n}{y}\PY{p}{,} \PY{n}{pred}\PY{p}{,} \PY{n}{th}\PY{p}{)}
         
             \PY{c+c1}{\PYZsh{} use TN and FN to compute NPV}
             \PY{n}{NPV} \PY{o}{=} \PY{n}{TN} \PY{o}{/} \PY{p}{(}\PY{n}{TN} \PY{o}{+} \PY{n}{FN}\PY{p}{)}
             
             \PY{c+c1}{\PYZsh{}\PYZsh{}\PYZsh{} END CODE HERE \PYZsh{}\PYZsh{}\PYZsh{}}
             
             \PY{k}{return} \PY{n}{NPV}
\end{Verbatim}


    \begin{Verbatim}[commandchars=\\\{\}]
{\color{incolor}In [{\color{incolor}31}]:} \PY{c+c1}{\PYZsh{}\PYZsh{}\PYZsh{} do not modify this cell    }
         \PY{n}{get\PYZus{}ppv\PYZus{}npv\PYZus{}test}\PY{p}{(}\PY{n}{get\PYZus{}ppv}\PY{p}{,} \PY{n}{get\PYZus{}npv}\PY{p}{)}    
\end{Verbatim}


    \begin{Verbatim}[commandchars=\\\{\}]
Test Case:

Test Labels:	   [1 0 0 1 1]
Test Predictions:  [1 0 0 1 1]
Threshold:	   0.5
Computed PPV:	   0.6666666666666666
Computed NPV:	   0.5 

\textcolor{ansi-green-intense}{ All tests passed.
}\textcolor{ansi-green-intense}{ All tests passed.
}
    \end{Verbatim}

    \hypertarget{expected-output}{%
\paragraph{Expected output:}\label{expected-output}}

\begin{Shaded}
\begin{Highlighting}[]
\NormalTok{Test Case:}

\NormalTok{Test Labels:       [}\DecValTok{1} \DecValTok{0} \DecValTok{0} \DecValTok{1} \DecValTok{1}\NormalTok{]}
\NormalTok{Test Predictions:  [}\DecValTok{1} \DecValTok{0} \DecValTok{0} \DecValTok{1} \DecValTok{1}\NormalTok{]}
\NormalTok{Threshold:       }\FloatTok{0.5}
\NormalTok{Computed PPV:     }\FloatTok{0.6666666666666666}
\NormalTok{Computed NPV:     }\FloatTok{0.5} 
\end{Highlighting}
\end{Shaded}

\begin{verbatim}
 All tests passed.
 All tests passed.
\end{verbatim}

    \begin{Verbatim}[commandchars=\\\{\}]
{\color{incolor}In [{\color{incolor}32}]:} \PY{n}{util}\PY{o}{.}\PY{n}{get\PYZus{}performance\PYZus{}metrics}\PY{p}{(}\PY{n}{y}\PY{p}{,} \PY{n}{pred}\PY{p}{,} \PY{n}{class\PYZus{}labels}\PY{p}{,} \PY{n}{acc}\PY{o}{=}\PY{n}{get\PYZus{}accuracy}\PY{p}{,} \PY{n}{prevalence}\PY{o}{=}\PY{n}{get\PYZus{}prevalence}\PY{p}{,} 
                                 \PY{n}{sens}\PY{o}{=}\PY{n}{get\PYZus{}sensitivity}\PY{p}{,} \PY{n}{spec}\PY{o}{=}\PY{n}{get\PYZus{}specificity}\PY{p}{,} \PY{n}{ppv}\PY{o}{=}\PY{n}{get\PYZus{}ppv}\PY{p}{,} \PY{n}{npv}\PY{o}{=}\PY{n}{get\PYZus{}npv}\PY{p}{)}
\end{Verbatim}


\begin{Verbatim}[commandchars=\\\{\}]
{\color{outcolor}Out[{\color{outcolor}32}]:}                      TP   TN   FP  FN Accuracy Prevalence Sensitivity  \textbackslash{}
                                                                                 
         Cardiomegaly         16  814  169   1     0.83      0.017       0.941   
         Emphysema            20  869  103   8    0.889      0.028       0.714   
         Effusion             99  690  196  15    0.789      0.114       0.868   
         Hernia                1  743  255   1    0.744      0.002         0.5   
         Infiltration        114  543  265  78    0.657      0.192       0.594   
         Mass                 40  789  158  13    0.829      0.053       0.755   
         Nodule               28  731  220  21    0.759      0.049       0.571   
         Atelectasis          64  657  249  30    0.721      0.094       0.681   
         Pneumothorax         24  785  183   8    0.809      0.032        0.75   
         Pleural\_Thickening   24  713  259   4    0.737      0.028       0.857   
         Pneumonia            14  661  320   5    0.675      0.019       0.737   
         Fibrosis             10  725  261   4    0.735      0.014       0.714   
         Edema                15  767  213   5    0.782       0.02        0.75   
         Consolidation        36  658  297   9    0.694      0.045         0.8   
         
                            Specificity    PPV    NPV          AUC           F1  \textbackslash{}
                                                                                  
         Cardiomegaly             0.828  0.086  0.999  Not Defined  Not Defined   
         Emphysema                0.894  0.163  0.991  Not Defined  Not Defined   
         Effusion                 0.779  0.336  0.979  Not Defined  Not Defined   
         Hernia                   0.744  0.004  0.999  Not Defined  Not Defined   
         Infiltration             0.672  0.301  0.874  Not Defined  Not Defined   
         Mass                     0.833  0.202  0.984  Not Defined  Not Defined   
         Nodule                   0.769  0.113  0.972  Not Defined  Not Defined   
         Atelectasis              0.725  0.204  0.956  Not Defined  Not Defined   
         Pneumothorax             0.811  0.116   0.99  Not Defined  Not Defined   
         Pleural\_Thickening       0.734  0.085  0.994  Not Defined  Not Defined   
         Pneumonia                0.674  0.042  0.992  Not Defined  Not Defined   
         Fibrosis                 0.735  0.037  0.995  Not Defined  Not Defined   
         Edema                    0.783  0.066  0.994  Not Defined  Not Defined   
         Consolidation            0.689  0.108  0.987  Not Defined  Not Defined   
         
                            Threshold  
                                       
         Cardiomegaly             0.5  
         Emphysema                0.5  
         Effusion                 0.5  
         Hernia                   0.5  
         Infiltration             0.5  
         Mass                     0.5  
         Nodule                   0.5  
         Atelectasis              0.5  
         Pneumothorax             0.5  
         Pleural\_Thickening       0.5  
         Pneumonia                0.5  
         Fibrosis                 0.5  
         Edema                    0.5  
         Consolidation            0.5  
\end{Verbatim}
            
    Notice that despite having very high sensitivity and accuracy, the PPV
of the predictions could still be very low.

This is the case with \texttt{Edema}, for example. * The sensitivity for
\texttt{Edema} is 0.75. * However, given that the model predicted
positive, the probability that a person has Edema (its PPV) is only
0.066!

     \#\#\# 3.6 ROC Curve

So far we have been operating under the assumption that our model's
prediction of \texttt{0.5} and above should be treated as positive and
otherwise it should be treated as negative. This however was a rather
arbitrary choice. One way to see this, is to look at a very informative
visualization called the receiver operating characteristic (ROC) curve.

The ROC curve is created by plotting the true positive rate (TPR)
against the false positive rate (FPR) at various threshold settings. The
ideal point is at the top left, with a true positive rate of 1 and a
false positive rate of 0. The various points on the curve are generated
by gradually changing the threshold.

Let's look at this curve for our model:

    \begin{Verbatim}[commandchars=\\\{\}]
{\color{incolor}In [{\color{incolor}33}]:} \PY{n}{util}\PY{o}{.}\PY{n}{get\PYZus{}curve}\PY{p}{(}\PY{n}{y}\PY{p}{,} \PY{n}{pred}\PY{p}{,} \PY{n}{class\PYZus{}labels}\PY{p}{)}
\end{Verbatim}


    \begin{center}
    \adjustimage{max size={0.9\linewidth}{0.9\paperheight}}{output_51_0.png}
    \end{center}
    { \hspace*{\fill} \\}
    
    The area under the ROC curve is also called AUCROC or C-statistic and is
a measure of goodness of fit. In medical literature this number also
gives the probability that a randomly selected patient who experienced a
condition had a higher risk score than a patient who had not experienced
the event. This summarizes the model output across all thresholds, and
provides a good sense of the discriminative power of a given model.

Let's use the \texttt{sklearn} metric function of
\texttt{roc\_auc\_score} to add this score to our metrics table.

    \begin{Verbatim}[commandchars=\\\{\}]
{\color{incolor}In [{\color{incolor}34}]:} \PY{k+kn}{from} \PY{n+nn}{sklearn}\PY{n+nn}{.}\PY{n+nn}{metrics} \PY{k}{import} \PY{n}{roc\PYZus{}auc\PYZus{}score}
         \PY{n}{util}\PY{o}{.}\PY{n}{get\PYZus{}performance\PYZus{}metrics}\PY{p}{(}\PY{n}{y}\PY{p}{,} \PY{n}{pred}\PY{p}{,} \PY{n}{class\PYZus{}labels}\PY{p}{,} \PY{n}{acc}\PY{o}{=}\PY{n}{get\PYZus{}accuracy}\PY{p}{,} \PY{n}{prevalence}\PY{o}{=}\PY{n}{get\PYZus{}prevalence}\PY{p}{,} 
                                 \PY{n}{sens}\PY{o}{=}\PY{n}{get\PYZus{}sensitivity}\PY{p}{,} \PY{n}{spec}\PY{o}{=}\PY{n}{get\PYZus{}specificity}\PY{p}{,} \PY{n}{ppv}\PY{o}{=}\PY{n}{get\PYZus{}ppv}\PY{p}{,} \PY{n}{npv}\PY{o}{=}\PY{n}{get\PYZus{}npv}\PY{p}{,} \PY{n}{auc}\PY{o}{=}\PY{n}{roc\PYZus{}auc\PYZus{}score}\PY{p}{)}
\end{Verbatim}


\begin{Verbatim}[commandchars=\\\{\}]
{\color{outcolor}Out[{\color{outcolor}34}]:}                      TP   TN   FP  FN Accuracy Prevalence Sensitivity  \textbackslash{}
                                                                                 
         Cardiomegaly         16  814  169   1     0.83      0.017       0.941   
         Emphysema            20  869  103   8    0.889      0.028       0.714   
         Effusion             99  690  196  15    0.789      0.114       0.868   
         Hernia                1  743  255   1    0.744      0.002         0.5   
         Infiltration        114  543  265  78    0.657      0.192       0.594   
         Mass                 40  789  158  13    0.829      0.053       0.755   
         Nodule               28  731  220  21    0.759      0.049       0.571   
         Atelectasis          64  657  249  30    0.721      0.094       0.681   
         Pneumothorax         24  785  183   8    0.809      0.032        0.75   
         Pleural\_Thickening   24  713  259   4    0.737      0.028       0.857   
         Pneumonia            14  661  320   5    0.675      0.019       0.737   
         Fibrosis             10  725  261   4    0.735      0.014       0.714   
         Edema                15  767  213   5    0.782       0.02        0.75   
         Consolidation        36  658  297   9    0.694      0.045         0.8   
         
                            Specificity    PPV    NPV    AUC           F1 Threshold  
                                                                                     
         Cardiomegaly             0.828  0.086  0.999  0.933  Not Defined       0.5  
         Emphysema                0.894  0.163  0.991  0.935  Not Defined       0.5  
         Effusion                 0.779  0.336  0.979  0.891  Not Defined       0.5  
         Hernia                   0.744  0.004  0.999  0.644  Not Defined       0.5  
         Infiltration             0.672  0.301  0.874  0.696  Not Defined       0.5  
         Mass                     0.833  0.202  0.984  0.888  Not Defined       0.5  
         Nodule                   0.769  0.113  0.972  0.745  Not Defined       0.5  
         Atelectasis              0.725  0.204  0.956  0.781  Not Defined       0.5  
         Pneumothorax             0.811  0.116   0.99  0.826  Not Defined       0.5  
         Pleural\_Thickening       0.734  0.085  0.994  0.868  Not Defined       0.5  
         Pneumonia                0.674  0.042  0.992  0.762  Not Defined       0.5  
         Fibrosis                 0.735  0.037  0.995  0.801  Not Defined       0.5  
         Edema                    0.783  0.066  0.994  0.856  Not Defined       0.5  
         Consolidation            0.689  0.108  0.987  0.799  Not Defined       0.5  
\end{Verbatim}
            
     \#\# 4. Confidence Intervals

    Of course our dataset is only a sample of the real world, and our
calculated values for all above metrics is an estimate of the real world
values. It would be good to quantify this uncertainty due to the
sampling of our dataset. We'll do this through the use of confidence
intervals. A 95\% confidence interval for an estimate \(\hat{s}\) of a
parameter \(s\) is an interval \(I = (a, b)\) such that 95\% of the time
when the experiment is run, the true value \(s\) is contained in \(I\).
More concretely, if we were to run the experiment many times, then the
fraction of those experiments for which \(I\) contains the true
parameter would tend towards 95\%.

While some estimates come with methods for computing the confidence
interval analytically, more complicated statistics, such as the AUC for
example, are difficult. For these we can use a method called the
\emph{bootstrap}. The bootstrap estimates the uncertainty by resampling
the dataset with replacement. For each resampling \(i\), we will get a
new estimate, \(\hat{s}_i\). We can then estimate the distribution of
\(\hat{s}\) by using the distribution of \(\hat{s}_i\) for our bootstrap
samples.

In the code below, we create bootstrap samples and compute sample AUCs
from those samples. Note that we use stratified random sampling
(sampling from the positive and negative classes separately) to make
sure that members of each class are represented.

    \begin{Verbatim}[commandchars=\\\{\}]
{\color{incolor}In [{\color{incolor}35}]:} \PY{k}{def} \PY{n+nf}{bootstrap\PYZus{}auc}\PY{p}{(}\PY{n}{y}\PY{p}{,} \PY{n}{pred}\PY{p}{,} \PY{n}{classes}\PY{p}{,} \PY{n}{bootstraps} \PY{o}{=} \PY{l+m+mi}{100}\PY{p}{,} \PY{n}{fold\PYZus{}size} \PY{o}{=} \PY{l+m+mi}{1000}\PY{p}{)}\PY{p}{:}
             \PY{n}{statistics} \PY{o}{=} \PY{n}{np}\PY{o}{.}\PY{n}{zeros}\PY{p}{(}\PY{p}{(}\PY{n+nb}{len}\PY{p}{(}\PY{n}{classes}\PY{p}{)}\PY{p}{,} \PY{n}{bootstraps}\PY{p}{)}\PY{p}{)}
         
             \PY{k}{for} \PY{n}{c} \PY{o+ow}{in} \PY{n+nb}{range}\PY{p}{(}\PY{n+nb}{len}\PY{p}{(}\PY{n}{classes}\PY{p}{)}\PY{p}{)}\PY{p}{:}
                 \PY{n}{df} \PY{o}{=} \PY{n}{pd}\PY{o}{.}\PY{n}{DataFrame}\PY{p}{(}\PY{n}{columns}\PY{o}{=}\PY{p}{[}\PY{l+s+s1}{\PYZsq{}}\PY{l+s+s1}{y}\PY{l+s+s1}{\PYZsq{}}\PY{p}{,} \PY{l+s+s1}{\PYZsq{}}\PY{l+s+s1}{pred}\PY{l+s+s1}{\PYZsq{}}\PY{p}{]}\PY{p}{)}
                 \PY{n}{df}\PY{o}{.}\PY{n}{loc}\PY{p}{[}\PY{p}{:}\PY{p}{,} \PY{l+s+s1}{\PYZsq{}}\PY{l+s+s1}{y}\PY{l+s+s1}{\PYZsq{}}\PY{p}{]} \PY{o}{=} \PY{n}{y}\PY{p}{[}\PY{p}{:}\PY{p}{,} \PY{n}{c}\PY{p}{]}
                 \PY{n}{df}\PY{o}{.}\PY{n}{loc}\PY{p}{[}\PY{p}{:}\PY{p}{,} \PY{l+s+s1}{\PYZsq{}}\PY{l+s+s1}{pred}\PY{l+s+s1}{\PYZsq{}}\PY{p}{]} \PY{o}{=} \PY{n}{pred}\PY{p}{[}\PY{p}{:}\PY{p}{,} \PY{n}{c}\PY{p}{]}
                 \PY{c+c1}{\PYZsh{} get positive examples for stratified sampling}
                 \PY{n}{df\PYZus{}pos} \PY{o}{=} \PY{n}{df}\PY{p}{[}\PY{n}{df}\PY{o}{.}\PY{n}{y} \PY{o}{==} \PY{l+m+mi}{1}\PY{p}{]}
                 \PY{n}{df\PYZus{}neg} \PY{o}{=} \PY{n}{df}\PY{p}{[}\PY{n}{df}\PY{o}{.}\PY{n}{y} \PY{o}{==} \PY{l+m+mi}{0}\PY{p}{]}
                 \PY{n}{prevalence} \PY{o}{=} \PY{n+nb}{len}\PY{p}{(}\PY{n}{df\PYZus{}pos}\PY{p}{)} \PY{o}{/} \PY{n+nb}{len}\PY{p}{(}\PY{n}{df}\PY{p}{)}
                 \PY{k}{for} \PY{n}{i} \PY{o+ow}{in} \PY{n+nb}{range}\PY{p}{(}\PY{n}{bootstraps}\PY{p}{)}\PY{p}{:}
                     \PY{c+c1}{\PYZsh{} stratified sampling of positive and negative examples}
                     \PY{n}{pos\PYZus{}sample} \PY{o}{=} \PY{n}{df\PYZus{}pos}\PY{o}{.}\PY{n}{sample}\PY{p}{(}\PY{n}{n} \PY{o}{=} \PY{n+nb}{int}\PY{p}{(}\PY{n}{fold\PYZus{}size} \PY{o}{*} \PY{n}{prevalence}\PY{p}{)}\PY{p}{,} \PY{n}{replace}\PY{o}{=}\PY{k+kc}{True}\PY{p}{)}
                     \PY{n}{neg\PYZus{}sample} \PY{o}{=} \PY{n}{df\PYZus{}neg}\PY{o}{.}\PY{n}{sample}\PY{p}{(}\PY{n}{n} \PY{o}{=} \PY{n+nb}{int}\PY{p}{(}\PY{n}{fold\PYZus{}size} \PY{o}{*} \PY{p}{(}\PY{l+m+mi}{1}\PY{o}{\PYZhy{}}\PY{n}{prevalence}\PY{p}{)}\PY{p}{)}\PY{p}{,} \PY{n}{replace}\PY{o}{=}\PY{k+kc}{True}\PY{p}{)}
         
                     \PY{n}{y\PYZus{}sample} \PY{o}{=} \PY{n}{np}\PY{o}{.}\PY{n}{concatenate}\PY{p}{(}\PY{p}{[}\PY{n}{pos\PYZus{}sample}\PY{o}{.}\PY{n}{y}\PY{o}{.}\PY{n}{values}\PY{p}{,} \PY{n}{neg\PYZus{}sample}\PY{o}{.}\PY{n}{y}\PY{o}{.}\PY{n}{values}\PY{p}{]}\PY{p}{)}
                     \PY{n}{pred\PYZus{}sample} \PY{o}{=} \PY{n}{np}\PY{o}{.}\PY{n}{concatenate}\PY{p}{(}\PY{p}{[}\PY{n}{pos\PYZus{}sample}\PY{o}{.}\PY{n}{pred}\PY{o}{.}\PY{n}{values}\PY{p}{,} \PY{n}{neg\PYZus{}sample}\PY{o}{.}\PY{n}{pred}\PY{o}{.}\PY{n}{values}\PY{p}{]}\PY{p}{)}
                     \PY{n}{score} \PY{o}{=} \PY{n}{roc\PYZus{}auc\PYZus{}score}\PY{p}{(}\PY{n}{y\PYZus{}sample}\PY{p}{,} \PY{n}{pred\PYZus{}sample}\PY{p}{)}
                     \PY{n}{statistics}\PY{p}{[}\PY{n}{c}\PY{p}{]}\PY{p}{[}\PY{n}{i}\PY{p}{]} \PY{o}{=} \PY{n}{score}
             \PY{k}{return} \PY{n}{statistics}
         
         \PY{n}{statistics} \PY{o}{=} \PY{n}{bootstrap\PYZus{}auc}\PY{p}{(}\PY{n}{y}\PY{p}{,} \PY{n}{pred}\PY{p}{,} \PY{n}{class\PYZus{}labels}\PY{p}{)}
\end{Verbatim}


    Now we can compute confidence intervals from the sample statistics that
we computed.

    \begin{Verbatim}[commandchars=\\\{\}]
{\color{incolor}In [{\color{incolor}36}]:} \PY{n}{util}\PY{o}{.}\PY{n}{print\PYZus{}confidence\PYZus{}intervals}\PY{p}{(}\PY{n}{class\PYZus{}labels}\PY{p}{,} \PY{n}{statistics}\PY{p}{)}
\end{Verbatim}


\begin{Verbatim}[commandchars=\\\{\}]
{\color{outcolor}Out[{\color{outcolor}36}]:}                    Mean AUC (CI 5\%-95\%)
         Cardiomegaly           0.93 (0.89-0.96)
         Emphysema              0.94 (0.92-0.96)
         Effusion               0.89 (0.87-0.91)
         Hernia                 0.63 (0.29-0.98)
         Infiltration           0.70 (0.67-0.73)
         Mass                   0.88 (0.84-0.92)
         Nodule                 0.74 (0.67-0.80)
         Atelectasis            0.78 (0.74-0.81)
         Pneumothorax           0.83 (0.76-0.88)
         Pleural\_Thickening     0.86 (0.80-0.91)
         Pneumonia              0.76 (0.67-0.83)
         Fibrosis               0.81 (0.75-0.86)
         Edema                  0.85 (0.81-0.90)
         Consolidation          0.80 (0.74-0.85)
\end{Verbatim}
            
    As you can see, our confidence intervals are much wider for some classes
than for others. Hernia, for example, has an interval around (0.30 -
0.98), indicating that we can't be certain it is better than chance (at
0.5).

     \#\# 5. Precision-Recall Curve

    Precision-Recall are informative prediction metrics when significant
class imbalance are present in the data.

In information retrieval - Precision is a measure of result relevancy
and that is equivalent to our previously defined PPV. - Recall is a
measure of how many truly relevant results are returned and that is
equivalent to our previously defined sensitivity measure.

The precision-recall curve (PRC) shows the trade-off between precision
and recall for different thresholds. A high area under the curve
represents both high recall and high precision, where high precision
relates to a low false positive rate, and high recall relates to a low
false negative rate.

High scores for both show that the classifier is returning accurate
results (high precision), as well as returning a majority of all
positive results (high recall).

Run the following cell to generate a PRC:

    \begin{Verbatim}[commandchars=\\\{\}]
{\color{incolor}In [{\color{incolor}37}]:} \PY{n}{util}\PY{o}{.}\PY{n}{get\PYZus{}curve}\PY{p}{(}\PY{n}{y}\PY{p}{,} \PY{n}{pred}\PY{p}{,} \PY{n}{class\PYZus{}labels}\PY{p}{,} \PY{n}{curve}\PY{o}{=}\PY{l+s+s1}{\PYZsq{}}\PY{l+s+s1}{prc}\PY{l+s+s1}{\PYZsq{}}\PY{p}{)}
\end{Verbatim}


    \begin{center}
    \adjustimage{max size={0.9\linewidth}{0.9\paperheight}}{output_62_0.png}
    \end{center}
    { \hspace*{\fill} \\}
    
     \#\# 6. F1 Score

    F1 score is the harmonic mean of the precision and recall, where an F1
score reaches its best value at 1 (perfect precision and recall) and
worst at 0.

Again, we can simply use \texttt{sklearn}'s utility metric function of
\texttt{f1\_score} to add this measure to our performance table.

    \begin{Verbatim}[commandchars=\\\{\}]
{\color{incolor}In [{\color{incolor}38}]:} \PY{k+kn}{from} \PY{n+nn}{sklearn}\PY{n+nn}{.}\PY{n+nn}{metrics} \PY{k}{import} \PY{n}{f1\PYZus{}score}
         \PY{n}{util}\PY{o}{.}\PY{n}{get\PYZus{}performance\PYZus{}metrics}\PY{p}{(}\PY{n}{y}\PY{p}{,} \PY{n}{pred}\PY{p}{,} \PY{n}{class\PYZus{}labels}\PY{p}{,} \PY{n}{acc}\PY{o}{=}\PY{n}{get\PYZus{}accuracy}\PY{p}{,} \PY{n}{prevalence}\PY{o}{=}\PY{n}{get\PYZus{}prevalence}\PY{p}{,} 
                                 \PY{n}{sens}\PY{o}{=}\PY{n}{get\PYZus{}sensitivity}\PY{p}{,} \PY{n}{spec}\PY{o}{=}\PY{n}{get\PYZus{}specificity}\PY{p}{,} \PY{n}{ppv}\PY{o}{=}\PY{n}{get\PYZus{}ppv}\PY{p}{,} \PY{n}{npv}\PY{o}{=}\PY{n}{get\PYZus{}npv}\PY{p}{,} \PY{n}{auc}\PY{o}{=}\PY{n}{roc\PYZus{}auc\PYZus{}score}\PY{p}{,}\PY{n}{f1}\PY{o}{=}\PY{n}{f1\PYZus{}score}\PY{p}{)}
\end{Verbatim}


\begin{Verbatim}[commandchars=\\\{\}]
{\color{outcolor}Out[{\color{outcolor}38}]:}                      TP   TN   FP  FN Accuracy Prevalence Sensitivity  \textbackslash{}
                                                                                 
         Cardiomegaly         16  814  169   1     0.83      0.017       0.941   
         Emphysema            20  869  103   8    0.889      0.028       0.714   
         Effusion             99  690  196  15    0.789      0.114       0.868   
         Hernia                1  743  255   1    0.744      0.002         0.5   
         Infiltration        114  543  265  78    0.657      0.192       0.594   
         Mass                 40  789  158  13    0.829      0.053       0.755   
         Nodule               28  731  220  21    0.759      0.049       0.571   
         Atelectasis          64  657  249  30    0.721      0.094       0.681   
         Pneumothorax         24  785  183   8    0.809      0.032        0.75   
         Pleural\_Thickening   24  713  259   4    0.737      0.028       0.857   
         Pneumonia            14  661  320   5    0.675      0.019       0.737   
         Fibrosis             10  725  261   4    0.735      0.014       0.714   
         Edema                15  767  213   5    0.782       0.02        0.75   
         Consolidation        36  658  297   9    0.694      0.045         0.8   
         
                            Specificity    PPV    NPV    AUC     F1 Threshold  
                                                                               
         Cardiomegaly             0.828  0.086  0.999  0.933  0.158       0.5  
         Emphysema                0.894  0.163  0.991  0.935  0.265       0.5  
         Effusion                 0.779  0.336  0.979  0.891  0.484       0.5  
         Hernia                   0.744  0.004  0.999  0.644  0.008       0.5  
         Infiltration             0.672  0.301  0.874  0.696  0.399       0.5  
         Mass                     0.833  0.202  0.984  0.888  0.319       0.5  
         Nodule                   0.769  0.113  0.972  0.745  0.189       0.5  
         Atelectasis              0.725  0.204  0.956  0.781  0.314       0.5  
         Pneumothorax             0.811  0.116   0.99  0.826  0.201       0.5  
         Pleural\_Thickening       0.734  0.085  0.994  0.868  0.154       0.5  
         Pneumonia                0.674  0.042  0.992  0.762  0.079       0.5  
         Fibrosis                 0.735  0.037  0.995  0.801   0.07       0.5  
         Edema                    0.783  0.066  0.994  0.856  0.121       0.5  
         Consolidation            0.689  0.108  0.987  0.799   0.19       0.5  
\end{Verbatim}
            
     \#\# 7. Calibration

    When performing classification we often want not only to predict the
class label, but also obtain a probability of each label. This
probability would ideally give us some kind of confidence on the
prediction. In order to observe how our model's generated probabilities
are aligned with the real probabilities, we can plot what's called a
\emph{calibration curve}.

In order to generate a calibration plot, we first bucketize our
predictions to a fixed number of separate bins (e.g.~5) between 0 and 1.
We then calculate a point for each bin: the x-value for each point is
the mean for the probability that our model has assigned to these points
and the y-value for each point fraction of true positives in that bin.
We then plot these points in a linear plot. A well-calibrated model has
a calibration curve that almost aligns with the y=x line.

The \texttt{sklearn} library has a utility \texttt{calibration\_curve}
for generating a calibration plot. Let's use it and take a look at our
model's calibration:

    \begin{Verbatim}[commandchars=\\\{\}]
{\color{incolor}In [{\color{incolor}39}]:} \PY{k+kn}{from} \PY{n+nn}{sklearn}\PY{n+nn}{.}\PY{n+nn}{calibration} \PY{k}{import} \PY{n}{calibration\PYZus{}curve}
         \PY{k}{def} \PY{n+nf}{plot\PYZus{}calibration\PYZus{}curve}\PY{p}{(}\PY{n}{y}\PY{p}{,} \PY{n}{pred}\PY{p}{)}\PY{p}{:}
             \PY{n}{plt}\PY{o}{.}\PY{n}{figure}\PY{p}{(}\PY{n}{figsize}\PY{o}{=}\PY{p}{(}\PY{l+m+mi}{20}\PY{p}{,} \PY{l+m+mi}{20}\PY{p}{)}\PY{p}{)}
             \PY{k}{for} \PY{n}{i} \PY{o+ow}{in} \PY{n+nb}{range}\PY{p}{(}\PY{n+nb}{len}\PY{p}{(}\PY{n}{class\PYZus{}labels}\PY{p}{)}\PY{p}{)}\PY{p}{:}
                 \PY{n}{plt}\PY{o}{.}\PY{n}{subplot}\PY{p}{(}\PY{l+m+mi}{4}\PY{p}{,} \PY{l+m+mi}{4}\PY{p}{,} \PY{n}{i} \PY{o}{+} \PY{l+m+mi}{1}\PY{p}{)}
                 \PY{n}{fraction\PYZus{}of\PYZus{}positives}\PY{p}{,} \PY{n}{mean\PYZus{}predicted\PYZus{}value} \PY{o}{=} \PY{n}{calibration\PYZus{}curve}\PY{p}{(}\PY{n}{y}\PY{p}{[}\PY{p}{:}\PY{p}{,}\PY{n}{i}\PY{p}{]}\PY{p}{,} \PY{n}{pred}\PY{p}{[}\PY{p}{:}\PY{p}{,}\PY{n}{i}\PY{p}{]}\PY{p}{,} \PY{n}{n\PYZus{}bins}\PY{o}{=}\PY{l+m+mi}{20}\PY{p}{)}
                 \PY{n}{plt}\PY{o}{.}\PY{n}{plot}\PY{p}{(}\PY{p}{[}\PY{l+m+mi}{0}\PY{p}{,} \PY{l+m+mi}{1}\PY{p}{]}\PY{p}{,} \PY{p}{[}\PY{l+m+mi}{0}\PY{p}{,} \PY{l+m+mi}{1}\PY{p}{]}\PY{p}{,} \PY{n}{linestyle}\PY{o}{=}\PY{l+s+s1}{\PYZsq{}}\PY{l+s+s1}{\PYZhy{}\PYZhy{}}\PY{l+s+s1}{\PYZsq{}}\PY{p}{)}
                 \PY{n}{plt}\PY{o}{.}\PY{n}{plot}\PY{p}{(}\PY{n}{mean\PYZus{}predicted\PYZus{}value}\PY{p}{,} \PY{n}{fraction\PYZus{}of\PYZus{}positives}\PY{p}{,} \PY{n}{marker}\PY{o}{=}\PY{l+s+s1}{\PYZsq{}}\PY{l+s+s1}{.}\PY{l+s+s1}{\PYZsq{}}\PY{p}{)}
                 \PY{n}{plt}\PY{o}{.}\PY{n}{xlabel}\PY{p}{(}\PY{l+s+s2}{\PYZdq{}}\PY{l+s+s2}{Predicted Value}\PY{l+s+s2}{\PYZdq{}}\PY{p}{)}
                 \PY{n}{plt}\PY{o}{.}\PY{n}{ylabel}\PY{p}{(}\PY{l+s+s2}{\PYZdq{}}\PY{l+s+s2}{Fraction of Positives}\PY{l+s+s2}{\PYZdq{}}\PY{p}{)}
                 \PY{n}{plt}\PY{o}{.}\PY{n}{title}\PY{p}{(}\PY{n}{class\PYZus{}labels}\PY{p}{[}\PY{n}{i}\PY{p}{]}\PY{p}{)}
             \PY{n}{plt}\PY{o}{.}\PY{n}{tight\PYZus{}layout}\PY{p}{(}\PY{p}{)}
             \PY{n}{plt}\PY{o}{.}\PY{n}{show}\PY{p}{(}\PY{p}{)}
\end{Verbatim}


    \begin{Verbatim}[commandchars=\\\{\}]
{\color{incolor}In [{\color{incolor}40}]:} \PY{n}{plot\PYZus{}calibration\PYZus{}curve}\PY{p}{(}\PY{n}{y}\PY{p}{,} \PY{n}{pred}\PY{p}{)}
\end{Verbatim}


    \begin{center}
    \adjustimage{max size={0.9\linewidth}{0.9\paperheight}}{output_69_0.png}
    \end{center}
    { \hspace*{\fill} \\}
    
    As the above plots show, for most predictions our model's calibration
plot does not resemble a well calibrated plot. How can we fix
that?\ldots{}

Thankfully, there is a very useful method called
\href{https://en.wikipedia.org/wiki/Platt_scaling}{Platt scaling} which
works by fitting a logistic regression model to our model's scores. To
build this model, we will be using the training portion of our dataset
to generate the linear model and then will use the model to calibrate
the predictions for our test portion.

    \begin{Verbatim}[commandchars=\\\{\}]
{\color{incolor}In [{\color{incolor}41}]:} \PY{k+kn}{from} \PY{n+nn}{sklearn}\PY{n+nn}{.}\PY{n+nn}{linear\PYZus{}model} \PY{k}{import} \PY{n}{LogisticRegression} \PY{k}{as} \PY{n}{LR} 
         
         \PY{n}{y\PYZus{}train} \PY{o}{=} \PY{n}{train\PYZus{}results}\PY{p}{[}\PY{n}{class\PYZus{}labels}\PY{p}{]}\PY{o}{.}\PY{n}{values}
         \PY{n}{pred\PYZus{}train} \PY{o}{=} \PY{n}{train\PYZus{}results}\PY{p}{[}\PY{n}{pred\PYZus{}labels}\PY{p}{]}\PY{o}{.}\PY{n}{values}
         \PY{n}{pred\PYZus{}calibrated} \PY{o}{=} \PY{n}{np}\PY{o}{.}\PY{n}{zeros\PYZus{}like}\PY{p}{(}\PY{n}{pred}\PY{p}{)}
         
         \PY{k}{for} \PY{n}{i} \PY{o+ow}{in} \PY{n+nb}{range}\PY{p}{(}\PY{n+nb}{len}\PY{p}{(}\PY{n}{class\PYZus{}labels}\PY{p}{)}\PY{p}{)}\PY{p}{:}
             \PY{n}{lr} \PY{o}{=} \PY{n}{LR}\PY{p}{(}\PY{n}{solver}\PY{o}{=}\PY{l+s+s1}{\PYZsq{}}\PY{l+s+s1}{liblinear}\PY{l+s+s1}{\PYZsq{}}\PY{p}{,} \PY{n}{max\PYZus{}iter}\PY{o}{=}\PY{l+m+mi}{10000}\PY{p}{)}
             \PY{n}{lr}\PY{o}{.}\PY{n}{fit}\PY{p}{(}\PY{n}{pred\PYZus{}train}\PY{p}{[}\PY{p}{:}\PY{p}{,} \PY{n}{i}\PY{p}{]}\PY{o}{.}\PY{n}{reshape}\PY{p}{(}\PY{o}{\PYZhy{}}\PY{l+m+mi}{1}\PY{p}{,} \PY{l+m+mi}{1}\PY{p}{)}\PY{p}{,} \PY{n}{y\PYZus{}train}\PY{p}{[}\PY{p}{:}\PY{p}{,} \PY{n}{i}\PY{p}{]}\PY{p}{)}    
             \PY{n}{pred\PYZus{}calibrated}\PY{p}{[}\PY{p}{:}\PY{p}{,} \PY{n}{i}\PY{p}{]} \PY{o}{=} \PY{n}{lr}\PY{o}{.}\PY{n}{predict\PYZus{}proba}\PY{p}{(}\PY{n}{pred}\PY{p}{[}\PY{p}{:}\PY{p}{,} \PY{n}{i}\PY{p}{]}\PY{o}{.}\PY{n}{reshape}\PY{p}{(}\PY{o}{\PYZhy{}}\PY{l+m+mi}{1}\PY{p}{,} \PY{l+m+mi}{1}\PY{p}{)}\PY{p}{)}\PY{p}{[}\PY{p}{:}\PY{p}{,}\PY{l+m+mi}{1}\PY{p}{]}
\end{Verbatim}


    \begin{Verbatim}[commandchars=\\\{\}]
{\color{incolor}In [{\color{incolor} }]:} \PY{n}{plot\PYZus{}calibration\PYZus{}curve}\PY{p}{(}\PY{n}{y}\PY{p}{[}\PY{p}{:}\PY{p}{,}\PY{p}{]}\PY{p}{,} \PY{n}{pred\PYZus{}calibrated}\PY{p}{)}
\end{Verbatim}


    \hypertarget{thats-it}{%
\section{That's it!}\label{thats-it}}

Congratulations! That was a lot of metrics to get familiarized with. We
hope that you feel a lot more confident in your understanding of medical
diagnostic evaluation and test your models correctly in your future work
:)


    % Add a bibliography block to the postdoc
    
    
    
    \end{document}
